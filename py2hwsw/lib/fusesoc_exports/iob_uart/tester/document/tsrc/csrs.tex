% SPDX-FileCopyrightText: 2025 IObundle, Lda
%
% SPDX-License-Identifier: MIT
%
% Py2HWSW Version 0.81 has generated this code (https://github.com/IObundle/py2hwsw).


The software accessible registers of the core are described in the following
tables. Each subsection corresponds to a specific configuration of the core, since
different configurations have different registers available.
The tables give information on the name, read/write capability, address, hardware and software width, and a
textual description. The addresses are byte aligned and given in hexadecimal format.
The hardware width is the number of bits that the register occupies in the hardware, while the
software width is the number of bits that the register occupies in the software.
In each address, the right-justified field having "Hw width" bits conveys the relevant information.
Each register has only one type of access, either read or write, meaning that reading from
a write-only register will produce invalid data or writing to a read-only register will
not have any effect.
\subsubsection{ Configuration}
\begin{xltabular}{\textwidth}{|l|c|c|c|c|c|X|}

  \hline
  \rowcolor{iob-green}
  {\bf Name} & {\bf R/W} & {\bf Addr} & \multicolumn{2}{c|}{\bf Width} & {\bf Default} & {\bf Description} \\ 
              &            &             & {\bf Hw}       & {\bf Sw}     &                &                    \\
  \hline

  \input _uart_csrs_tab

  \caption{UART software accessible registers}
\end{xltabular}
\label{_uart_csrs_tab:is}

\begin{xltabular}{\textwidth}{|l|c|c|c|c|c|X|}

  \hline
  \rowcolor{iob-green}
  {\bf Name} & {\bf R/W} & {\bf Addr} & \multicolumn{2}{c|}{\bf Width} & {\bf Default} & {\bf Description} \\ 
              &            &             & {\bf Hw}       & {\bf Sw}     &                &                    \\
  \hline

  \input _general_operation_csrs_tab

  \caption{General Registers.}
\end{xltabular}
\label{_general_operation_csrs_tab:is}
