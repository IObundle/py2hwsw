% SPDX-FileCopyrightText: 2025 IObundle
%
% SPDX-License-Identifier: MIT

This section provides three end-to-end examples of using Py2HWSW to generate and verify digital hardware cores. The examples cover the iob\_aoi, iob\_pulse\_gen, and iob\_soc cores, showcasing the automation of Verilog module generation from attributes.

\subsubsection{iob\_aoi Example}

The iob\_aoi core is a simple example that combines an AND, OR, and invert logic gates. The core's attributes are defined in the iob\_aoi.py file, available at \url{https://github.com/IObundle/py2hwsw/tree/main/py2hwsw/lib/hardware/basic_tests/iob_aoi}. To generate the iob\_aoi core, follow these steps:

\begin{enumerate}
\item Create or modify the iob\_aoi.py file to set the attributes of the core, describing how it should be generated using the Py2HWSW standard core dictionary interface.
\item Optionally, add more files to the setup directory as needed, such as manual Verilog sources or templates, scripts, or software.
\item Call the Py2HWSW setup process using the command:
\begin{lstlisting}[language=bash]
nix-shell --run "py2hwsw iob_aoi setup --build_dir '$(BUILD_DIR)'"
\end{lstlisting}
\item The generated build directory contains all the Verilog sources, Makefile, and configurations to run the core in various flows (simulation, FPGA).
\item To run the core in simulation, call the command: \texttt{make sim-run} from the build directory.
\end{enumerate}

\subsubsection{iob\_pulse\_gen Example}

The iob\_pulse\_gen core is used to generate wire pulses with configurable start and duration. The core's attributes are defined in the iob\_pulse\_gen.py file, available at \url{https://github.com/IObundle/py2hwsw/blob/main/py2hwsw/lib/hardware/clocks_resets/iob_pulse_gen/iob_pulse_gen.py}. To generate the iob\_pulse\_gen core, follow these steps:

\begin{enumerate}
\item Create or modify the iob\_pulse\_gen.py file to set the attributes of the core, describing how it should be generated using the Py2HWSW standard core dictionary interface.
\item Optionally, add more files to the setup directory as needed, such as manual Verilog sources or templates, scripts, or software.
\item Call the Py2HWSW setup process using the command:
\begin{lstlisting}[language=bash]
nix-shell --run "py2hwsw iob_pulse_gen setup --build_dir '$(BUILD_DIR)'"
\end{lstlisting}
\item The generated build directory contains all the Verilog sources, Makefile, and configurations to run the core in various flows (simulation, FPGA).
\item To run the core in simulation, call the command: \texttt{make sim-run} from the build directory.
\end{enumerate}

\subsubsection{iob\_soc Example}

The iob\_soc core is a more complex example used to create a system on chip. The core's attributes are defined in the iob\_soc.py file, available at \url{https://github.com/IObundle/iob-soc}. The iob\_soc.py file supports high-level Python parameters that allow configuring main SoC components like the CPU, memories, and peripherals. To generate the iob\_soc core, follow these steps:

\begin{enumerate}
\item Create or modify the iob\_soc.py file to set the attributes of the core, describing how it should be generated using the Py2HWSW standard core dictionary interface.
\item Optionally, add more files to the setup directory as needed, such as manual Verilog sources or templates, scripts, or software.
\item Call the Py2HWSW setup process using the command:
\begin{lstlisting}[language=bash]
nix-shell --run "py2hwsw iob_soc setup --build_dir '$(BUILD_DIR)'"
\end{lstlisting}
\item The generated build directory contains all the Verilog sources, Makefile, and configurations to run the core in various flows (simulation, FPGA).
\item To run the core in simulation, call the command: \texttt{make sim-run} from the build directory.
\end{enumerate}

These examples demonstrate the automation of Verilog module generation from attributes using Py2HWSW, showcasing the flexibility and ease of use of the framework.
