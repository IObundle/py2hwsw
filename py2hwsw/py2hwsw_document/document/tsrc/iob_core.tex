% SPDX-FileCopyrightText: 2025 IObundle
%
% SPDX-License-Identifier: MIT

The \href{https://github.com/IObundle/py2hwsw/blob/main/py2hwsw/scripts/iob_core.py}{iob\_core} class is a central component of the Py2HWSW framework, responsible for representing and managing IP cores. The class provides a structured way to describe and generate IP cores, making it easier to create and integrate complex digital designs. At the heart of the iob\_core class is its constructor, which plays a crucial role in setting up and initializing the core.

The constructor of the iob\_core class is responsible for converting and initializing the attributes of the core, setting up subblocks, and generating the sources for the current core in the build directory. When the constructor is called, it distinguishes between two situations: when it is the top module, it creates a build directory for users with all the dependencies and project flows, and when it is a subblock, it provides all the information necessary for instantiation and integration. The constructor takes care of setting up the build environment, generating Verilog code, and creating the build directory, making it a key component of the Py2HWSW framework.

The iob\_core constructor also handles the setup process for subblocks and superblocks. When a subblock is encountered, the constructor is called recursively to set up the subblock's attributes and generate its sources. Similarly, when a superblock is encountered, the constructor sets up the superblock's attributes and generates its sources, ensuring that the entire hierarchy of modules and subblocks is properly initialized and configured, as detailed in Section~\ref{sec:py2_block_hierarchy}

Overall, the iob\_core class and its constructor provide a powerful and flexible way to represent and manage IP cores, making it a fundamental component of the Py2HWSW framework.

It inherits attributes from its parent classes \href{https://github.com/IObundle/py2hwsw/blob/main/py2hwsw/scripts/iob_module.py}{iob\_module} and \href{https://github.com/IObundle/py2hwsw/blob/main/py2hwsw/scripts/iob_instance.py}{iob\_instance}.

\lstinputlisting[firstline=49,lastline=49,language=Python]{iob_core.py}
\href{https://github.com/IObundle/py2hwsw/blob/main/py2hwsw/scripts/iob_core.py}{View Source}

% get_core_obj function

The \href{https://github.com/IObundle/py2hwsw/blob/main/py2hwsw/scripts/iob_core.py#L858}{get\_core\_obj} function is used to generate an instance of a core based on a given core name and python parameters.
This method will search for the corresponding Python or JSON file of the core, and generate a python object based on info stored in that file, and info passed via python parameters.

\lstinputlisting[firstline=858,lastline=906,language=Python]{iob_core.py}
\href{https://github.com/IObundle/py2hwsw/blob/main/py2hwsw/scripts/iob_core.py}{View Source}
