% SPDX-FileCopyrightText: 2025 IObundle
%
% SPDX-License-Identifier: MIT

To set up a core with Py2HWSW, you'll need to have Nix installed on your system. You can download and install Nix from the official Nix website. Once Nix is installed, you can clone the Py2HWSW repository using the command \texttt{git clone --recursive git@github.com:IObundle/py2hwsw.git}.

Next, navigate to the Py2HWSW directory and run the command \texttt{nix-shell} to enter the Nix-shell environment. This will ensure that all dependencies required by Py2HWSW are installed and available.

To set up a core, you can use the command:
\begin{lstlisting}[language=bash]
nix-shell --run "py2hwsw $(CORE) setup --build_dir '$(BUILD_DIR)' --iob_params 'param1=param1_val:param2=param2_val'"
\end{lstlisting}

This command will generate the necessary files and directories for your core in the specified build directory.

You can customize the setup process by passing additional options to the \texttt{py2hwsw} command. For example, you can disable format and linting checks by adding the options \texttt{--no\_verilog\_lint} and \texttt{--no\_verilog\_format} to the command.

Here's an example of a setup directory structure:
\begin{verbatim}
mycore/
  mycore.py
  hardware/
    src/
      mycore.v
\end{verbatim}
In this example, the \texttt{mycore.py} file contains the core description, and the \texttt{hardware/src} directory contains the Verilog source files for the core.

In some cases, the Verilog source file (\texttt{mycore.v}) may not be necessary, as the \texttt{mycore.py} file can describe the entire core, including its ports, buses, components, and even custom Verilog code. This allows for a high degree of flexibility and customization, as users can define their core's architecture and behavior entirely within the Python description file.

The Python description is particularly useful because it enables the creation of higher-level abstractions, making it easier to design and work with complex hardware components. Additionally, the use of IOb parameters allows for dynamic modification of cores, enabling users to easily customize and adapt their designs to different use cases and applications.

To set up this core, you would run the command:
\begin{lstlisting}[language=bash]
nix-shell --run "py2hwsw mycore setup --build_dir './build' --iob_params 'param1=param1_val:param2=param2_val'"
\end{lstlisting}

This would generate the necessary files and directories for the core in the \texttt{./build} directory.

Note that you can customize the setup process to fit your specific needs by modifying the core description, Verilog source files, and setup command options.

