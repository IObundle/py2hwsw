% SPDX-FileCopyrightText: 2025 IObundle
%
% SPDX-License-Identifier: MIT

The Py2HWSW framework provides the following two standard interfaces:
\begin{enumerate}
  \item \textbf{Python Parameters}: Core "setup" function receives information from Py2HWSW via a dictionary in its first argument, referred to as \textit{Python Parameters}.
  \item \textbf{Core Dictionary}: Core "setup" function returns a core description dictionary to Py2HWSW, referred to as \textit{Core Dictionary}.
\end{enumerate}

The core's "setup" function is the python function defined by the user in the <core\_name>.py file.

If the core is described by a JSON file, then the \textit{Python Parameters} interface is not available.
The JSON file gives a dictionary to Py2HWSW, similar to the python dictionary of the "setup" function.
This allows the user to use external tools to generate cores in JSON format.

%
% Python parameters
%

The \textit{Python Parameters} received by the core's "setup" function is a dictionary containing both parameters passed by its issuer and standard parameters passed by Py2HWSW.
Each key, value pair in the dictionary is a \textit{Python Parameter}.
The value of the python parameter may be of any data type.

\begin{xltabular}{\textwidth}{|l|l|X|}

  \hline
  \rowcolor{iob-green}
  {\bf Name} & {\bf Data Type} & {\bf Description}  \\ \hline \hline

  \input py2hwsw_py_params_tab

  \caption{Standard \textit{Python Parameters} passed by Py2HWSW to every core's "setup" function.}
\end{xltabular}
\label{py2hwsw_py_params_tab}

The standard python parameters passed by Py2HWSW are listed in Table~\ref{py2hwsw_py_params_tab}.

The python parameters supported by each core is available in the respective core's user guide, as long as they have the \textit{Python Parameters} attribute defined.
Instructions on how to build a core's user guide can be found in Section~\ref{sec:core_lib}. 


%
% Core dictionary
%

\begin{xltabular}{\textwidth}{|l|l|X|}
  \hline
  \rowcolor{iob-green}
  % TODO: The "Data Type" column should specify what the user should input, instead of the internal object used by py2hwsw.
  {\bf Name} & {\bf Data Type} & {\bf Description}  \\ \hline \hline
  \endhead

  \input py2hwsw_attributes_tab

  \caption{Table of supported Py2HWSW attributes in the \textbf{Core Dictionary}. The \textit{Data Type} column specifies the type of internal object that the Py2HWSW will convert the attribute's value to (usually the user inputs a string, list, or dictionary value and then py2 converts it to an internal object).}
  \label{py2hwsw_attributes_tab}
\end{xltabular}

The list of attributes supported by the Py2HWSW framework is given in Table~\ref{py2hwsw_attributes_tab}.
If a core provides a dictionary with keys not listed in Table~\ref{py2hwsw_attributes_tab}, then the Py2HWSW framework will raise an error.
Each key, value pair in the dictionary is a \textit{Core Attribute}.
The data type of the core attribute may be of any data type, but are usually a string, list, or dictionary.
If the data type is a string, it may also represent an object using Py2HWSW's \textit{Short Notation}.
%~\ref{sec:short_notation}
