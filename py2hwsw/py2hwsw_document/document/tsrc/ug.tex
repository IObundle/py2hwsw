% SPDX-FileCopyrightText: 2025 IObundle
%
% SPDX-License-Identifier: MIT

\documentclass{ug}

%
% Packages and configuration
%

\usepackage{xltabular}

\usepackage{hyperref}
% Set up hyperlink colors
\hypersetup{
    colorlinks=true, % false: boxed links; true: colored links
    linkcolor=blue, % color of internal links
    citecolor=blue, % color of citations
    urlcolor=blue   % color of external links
}


\usepackage{listings}
\usepackage{xcolor}
% Define colors
\definecolor{keywordcolor}{RGB}{0, 0, 255} % Blue for keywords
\definecolor{commentcolor}{RGB}{0, 128, 0} % Green for comments
\definecolor{stringcolor}{RGB}{255, 0, 0} % Red for strings
\definecolor{backgroundcolor}{RGB}{245, 245, 245} % Light gray background

% Set up listings
\lstdefinestyle{Python}{
    backgroundcolor=\color{backgroundcolor},
    basicstyle=\ttfamily,
    keywordstyle=\color{keywordcolor},
    commentstyle=\color{commentcolor},
    stringstyle=\color{stringcolor},
    numbers=left,
    numberstyle=\tiny,
    stepnumber=1,
    numbersep=5pt,
    showstringspaces=false,
    tabsize=4,
    breaklines=true
}
% Set the default style for all lstlisting environments
\lstset{style=Python}

% Support unicode chars for drawing directory trees (├ ─ └)
\usepackage{pmboxdraw}

%
% Document content
%

% SPDX-FileCopyrightText: 2024 IObundle
%
% SPDX-License-Identifier: MIT

%replace ipcore-name by the name of your ip core (e.g. IOb-Cache) and description by a brief description (e.g. a Configurable Cache)

\title{%
\Huge IOb-System \\
 \vspace*{3cm}
\Large System-on-Chip (SoC) template 
}

\header{IOb-System, System-on-Chip (SoC) template}

\date{\today}
\category{User Guide, \input{\NAME_version.tex}, Build \input{shortHash.tex}}

\input{color}

\begin{document}

\maketitle
\pagenumbering{gobble}

\vspace*{\fill}
User Guide, \input{\NAME_version.tex}, Build \input{shortHash.tex}
\hspace*{\fill} \includegraphics[keepaspectratio,scale=.7]{Logo.png}

\cleardoublepage
\pagenumbering{roman}
\setcounter{page}{1}
% SPDX-FileCopyrightText: 2025 IObundle
%
% SPDX-License-Identifier: MIT

\today & Initial document version \input{version}. \\ \hline

\cleardoublepage
\tableofcontents
\clearpage
\listoftables
\clearpage
\listoffigures
\cleardoublepage
\pagenumbering{arabic}
\setcounter{page}{1}

%
% Introduction
%
\section{Introduction}
\label{sec:intro}
% SPDX-FileCopyrightText: 2024 IObundle
%
% SPDX-License-Identifier: MIT

\lipsum[1-1]

%% SPDX-FileCopyrightText: 2025 IObundle
%
% SPDX-License-Identifier: MIT

\begin{figure}[!htbp]
  \centerline{\includegraphics[width=14cm]{symb.pdf}}
  \vspace{0cm}
  \caption{IP Core Symbol}
  \label{fig:symbol}
\end{figure}


\subsection{What Is Py2HWSW?}
\label{sec:whatispy2}
% SPDX-FileCopyrightText: 2024 IObundle
%
% SPDX-License-Identifier: MIT

In the rapidly evolving landscape of hardware design, the need for efficient and flexible tools is paramount. Enter py2hwsw, a powerful tool designed to streamline the process of generating Verilog cores from high-level descriptions provided in Python or JSON dictionaries. With py2hwsw, engineers can easily translate their design specifications into functional hardware components, significantly reducing development time and complexity.


\subsection{What Is Py2HWSW For?}
\label{sec:purpose}
% SPDX-FileCopyrightText: 2025 IObundle
%
% SPDX-License-Identifier: MIT

Py2HWSW is designed to do the following:
\begin{itemize}
    \item \textbf{Core Generation}: Generates Verilog cores from descriptions in Python or JSON dictionaries.

    \item \textbf{Framework Compatibility}: Integrates seamlessly with existing Verilog cores and frameworks.

    \item \textbf{High-Level Configuration}: Allows configuration of cores via high-level Python parameters.

    \item \textbf{Automated Resources}: Produces scripts and Makefiles for deployment in various FPGAs, simulators, and synthesis tools, along with documentation.

    \item \textbf{Readable Code}: Generates legible Verilog code with comments for better understanding and maintenance.

\end{itemize}


\subsection{What Problem Does Py2HWSW Solve?}
\label{sec:problem}
% SPDX-FileCopyrightText: 2024 IObundle
%
% SPDX-License-Identifier: MIT

Py2hwsw addresses several key challenges in the hardware design process:
\begin{itemize}
    \item \textbf{Complexity of Verilog Coding}: Writing Verilog code can be intricate and error-prone, especially for those who may not be deeply familiar with hardware description languages. Py2hwsw simplifies this by allowing designers to specify their hardware requirements using high-level Python or JSON dictionaries, reducing the need for extensive Verilog knowledge.

    \item \textbf{Integration of Existing Designs}: Many projects involve legacy Verilog cores that need to be integrated with new designs. Py2hwsw facilitates this integration, enabling users to leverage existing components while still benefiting from the tool's advanced features.

    \item \textbf{Configuration Challenges}: Customizing hardware components often requires deep dives into low-level code. Py2hwsw allows for high-level configuration through Python parameters, making it easier for designers to adjust their designs without getting bogged down in the details of Verilog.

    \item \textbf{Resource Generation}: The process of preparing scripts and Makefiles for various deployment environments can be tedious and time-consuming. Py2hwsw automates this process, providing users with the necessary resources to run their designs on different FPGAs, simulators, and synthesis tools.

    \item \textbf{Code Readability and Maintenance}: Maintaining and debugging hardware designs can be challenging, especially when the code is not well-documented. Py2hwsw generates legible Verilog code with comments, enhancing readability and making it easier for teams to collaborate and maintain their designs over time.
\end{itemize}

In summary, Py2hwsw streamlines the hardware design workflow, making it more accessible, efficient, and manageable for engineers and designers.


\subsection{What Design Principles Underlie Py2HWSW?}
\label{sec:principles}
% SPDX-FileCopyrightText: 2024 IObundle
%
% SPDX-License-Identifier: MIT

Py2HWSW works by:
* TODO: Describe design principles here


\subsection{How Does Py2HWSW Accomplish Its Goals?}
\label{sec:how}
% SPDX-FileCopyrightText: 2024 IObundle
%
% SPDX-License-Identifier: MIT

Py2HWSW does this by:
\begin{itemize}
\item \textbf{Two-Step Development Process}: The core development is divided
into two distinct phases: the \textbf{setup} phase and the \textbf{build}
phase. During the setup phase, Verilog source files are generated based on
high-level descriptions provided in Python or JSON format. The build phase then
utilizes these Verilog sources to produce the necessary FPGA bitstreams,
netlists, and other deployment files.

\item \textbf{Independent Setup Folders}: Each core is organized within its own
independent setup folder, containing high-level description files and, if
needed, low-level files as well.

\item \textbf{Core Description Input}: The core's specifications are provided
to Py2hwsw in the form of JSON or a Python dictionary, utilizing standard
Py2hwsw attributes.

\item \textbf{Flexible Attribute Handling}: When generating the cores
dictionary via a Python script, users can include a set of standard Py2hwsw
attributes alongside their own custom-defined attributes.
\end{itemize}

Learn more about [[How It Works]] and [[How To Use]].


%
% Getting Started
%
\ifdefined\SECTIONCLEARPAGE
\clearpage
\fi
\section{Getting Started}
\label{sec:gs}

\subsection{Setup Directory}
\label{sec:setup_dir}
The setup directory of a core may have the following structure:

```
.
├── core_name.py
├── core_name.json
├── document
│   ├── doc_build.mk
│   ├── figures
│   └── tsrc
├── hardware
│   ├── src
│   ├── fpga
│   │   ├── fpga_build.mk
│   │   ├── src
│   │   ├── quartus
│   │   └── vivado
│   ├── modules
│   ├── simulation
│   │   ├── sim_build.mk
│   │   └── src
│   └── syn
│       ├── src
│       └── genus
├── software
│   ├── sw_build.mk
│   └── src
├── scripts
├── submodules
├── Makefile
├── README.md
├── LICENSE
├── CITATION.cff
└── default.nix
```

Only the `core_name.py` or `core_name.json` file is needed to pass the core's description to Py2HWSW.
The remaining directories and files are optional.

If the `document`, `hardware`, and `software` directories exist, they will be copied to the `build` directory, overriding any files already present there, such as standard ones or files from other cores.

The `*_build.mk` files allow the user to include core specific Makefile targets and variables from the build process.
These will be copied to the `build` directory and included in the standard build process Makefiles.

The `src` directories contain manually written Verilog/C/TeX sources for the core, should they be needed.


The following directories and files do not follow a mandatory structure, but are typically used for the following purposes:

The `hardware/modules` and `submodules` directories typically contain setup directories of other cores.

The `scripts` directory contains scripts specific to the core, and may be called by the user or from the `core_name.py` script.


\subsection{Create An AND Gate Core: iob\_and}
\label{sec:iob_and}
% SPDX-FileCopyrightText: 2025 IObundle
%
% SPDX-License-Identifier: MIT

The simplest core description for Py2HWSW is as follows:

% py2_macro: file iob_and.py

A set of basic cores to showcase the various Py2HWSW features can be found in
the \href{https://github.com/IObundle/py2hwsw/tree/main/py2hwsw/lib/hardware/basic_tests}{basic\_tests}
directory.


\subsection{Setup And Build}
\label{sec:setup_build}
% SPDX-FileCopyrightText: 2024 IObundle
%
% SPDX-License-Identifier: MIT

To checkout the source and setup the core:

\begin{verbatim}
$ git clone --recursive git@github.com:IObundle/iob-soc.git
$ cd iob-soc
$ nix-shell  # Optional step to install environment with necessary dependencies
$ py2hwsw iob_soc setup --no_verilog_lint --py_params 'init_mem=1:use_extmem=0'
\end{verbatim}

To do a clean setup:

\begin{verbatim}
$ py2hwsw iob_soc clean
$ py2hwsw iob_soc setup --no_verilog_lint --py_params 'init_mem=1:use_extmem=0'
\end{verbatim}

The setup process will generate a build directory containing the core's verilog sources and build files.
By default, the build directory is `../[core\_name]\_V[core\_version]`.

To build and run the core in simulation:
\begin{verbatim}
$ make -C ../iob_soc_V0.7 sim-run
\end{verbatim}


%
% How It Works
%
\ifdefined\SECTIONCLEARPAGE
\clearpage
\fi
\section{How It Works}
\label{sec:how_it_works}

This section gives a detailed description of the Py2HWSW framework.

\subsection{Setup Flow Chart}
\label{sec:py2_flow_chart}

Figure~\ref{fig:py2_flow_chart} presents a high-level flow chart of the Py2HWSW setup procedure.

\begin{figure}[H]
  \centering {\includegraphics[width=\columnwidth]{py2_flow_chart.pdf}}
  \vspace{-0.7cm}
  \caption{High-Level Flow Chart of Py2HWSW Setup Procedure}
  \label{fig:py2_flow_chart}
\end{figure}

\subsection{Standard Interfaces}
\label{sec:py2_standard_interfaces}
% SPDX-FileCopyrightText: 2024 IObundle
%
% SPDX-License-Identifier: MIT

The py2hwsw framework provides the following two standard interfaces:
1) Core "setup" function receives information from py2hwsw via "Python Parameters".
2) Core "setup" function returns a core description to py2hwsw via a python dictionary.

The core's "setup" function is the python function defined by the user in the <core_name>.py file.

If the core is described by a JSON file, then the "Python Parameters" interface is not available.
The JSON file gives a dictionary to py2hwsw, similar to the python dictionary of the "setup" function.
This allows the user to use external tools to generate cores in JSON format.




\subsection{Block hierarchy}
\label{sec:py2_block_hierarchy}

Figure~\ref{fig:py2_superblocks_subblocks} presents an example block hierarchy for a Py2HWSW project.
Superblocks are only used if they are superblocks of the project's top module or of one of its wrappers.

\begin{figure}[H]
  \centering {\includegraphics[width=\columnwidth]{superblocks_subblocks.pdf}}
  \vspace{-0.7cm}
  \caption{Block Hierarchy of a Py2HWSW Project}
  \label{fig:py2_superblocks_subblocks}
\end{figure}

\subsection{Main launch script: py2hwsw.py}
\label{sec:launch_script}
% SPDX-FileCopyrightText: 2025 IObundle
%
% SPDX-License-Identifier: MIT

The main launch script for the Py2HWSW progam is the `py2hwsw.py` script.

The following code snippet from that script processes the command line arguments and launches the program for the specified "target".

\lstinputlisting[firstline=106,language=Python]{py2hwsw.py}
\href{https://github.com/IObundle/py2hwsw/blob/main/py2hwsw/scripts/py2hwsw.py}{View Source}




\subsection{Simulate with Verilator}
\label{sec:verilator}
% SPDX-FileCopyrightText: 2025 IObundle
%
% SPDX-License-Identifier: MIT

With mandatory structured IOs, the testbench behaves like a processor reading
and writing to its CSR. A universal Verilator testbench has been developed for
an IP with a structured IOb native interface (bridges to standard AXI-Lite, APB
or Wishbone are supplied). The testbench is a C++ program provides hardware
reset and CSR read and write functions.

\subsubsection{IP core simulation}

The IP cores using this testbench must provide a C function called
{\tt iob\_core\_tb()}, the IP core’s specific test. They also must provide a C header
called {\tt iob\_vlt\_tb.h} that defines the Device Under Test (DUT) as a Verilator
type called {\tt dut\_t}. With knowledge of the DUT and its test, the universal
Verilator testbench will exercise any IP core. Interestingly, {\tt iob\_core\_tb()} also
runs, without modifications, on a RISC-V processor with the IP as a submodule,
for example, for FPGA testing or emulation.

The {\tt iob\_uart} core is used as an example, located in the {\tt
  py2hwsw/lib/peripherals/iob\_uartiob-uart} directory.

\begin{lstlisting}[language=bash]
  $ git clone --recursive git@github.com:IObundle/py2hwsw.git
  $ cd py2hwsw/lib
  $ make sim-run CORE=iob\_uart SIMULATOR=verilator
\end{lstlisting}

The {\tt make sim-run} command will run core setup, creating the build directory
at {\tt ../../../iob\_uart\_V0.1}. The Verilator simulator will be run in the
build directory. The testbench will be compiled and run, and the output will be
displayed on the console.

\subsubsection{Subsystem simulation}

To illustrate system test capabilities with the universal Verilator testbench,
the {\tt iob\_system} subsystem core is used as an example, located in the {\tt
  py2hwsw/lib/iob\_system} directory.

\begin{lstlisting}[language=bash]
  $ git clone --recursive git@github.com:IObundle/py2hwsw.git
  $ cd py2hwsw/lib
  $ make sim-run CORE=iob\_uart SIMULATOR=verilator
\end{lstlisting}

In this case the {\tt iob\_core\_tb()} function is running on the desktop,
emualting a system tester. The console output comes from teh system itself
runnig its embedded test, a more elaborated form of a hello world program.


\subsection{Deliver an IP core}
\label{sec:deliver}
% SPDX-FileCopyrightText: 2025 IObundle
%
% SPDX-License-Identifier: MIT

From the build directory, we select the essential files to create a tarball, all
containing a Makefile-driven environment for the user who, in this way, will not
need any ancillary tools beyond the standard EDA tools.

\begin{lstlisting}[language=bash]
  $ git clone --recursive git@github.com:IObundle/py2hwsw.git
  $ cd py2hwsw/
  $ nix-shell py2hwsw/lib/ # Optional step to install environment with necessary dependencies
  $ py2hwsw iob_uart setup --no_verilog_lint
  $ py2hwsw iob_uart deliver
\end{lstlisting}

The tarball wil be created in the \texttt{../iob\_uart\_V0.1} directory, which is
also the home of the default build directory.

%
% Py2HWSW classes
%
\section{Py2HWSW Classes}
\label{sec:py_classes}

\subsection{Main class for core representation: iob\_core.py}
\label{sec:iob_core}
% SPDX-FileCopyrightText: 2025 IObundle
%
% SPDX-License-Identifier: MIT

The \href{https://github.com/IObundle/py2hwsw/blob/main/py2hwsw/scripts/iob_core.py}{iob\_core} class is the main class used to represent a core.

It inherits attributes from its parent classes \href{https://github.com/IObundle/py2hwsw/blob/main/py2hwsw/scripts/iob_module.py}{iob\_module} and \href{https://github.com/IObundle/py2hwsw/blob/main/py2hwsw/scripts/iob_instance.py}{iob\_instance}.

\lstinputlisting[firstline=49,lastline=49,language=Python]{iob_core.py}
\href{https://github.com/IObundle/py2hwsw/blob/main/py2hwsw/scripts/iob_core.py}{View Source}

The \href{https://github.com/IObundle/py2hwsw/blob/main/py2hwsw/scripts/iob_core.py#L858}{get\_core\_obj} function is used to generate an instance of a core based on a given core name and python parameters.
This method will search for the corresponding Python or JSON file of the core, and generate a python object based on info stored in that file, and info passed via python parameters.

\lstinputlisting[firstline=858,lastline=906,language=Python]{iob_core.py}
\href{https://github.com/IObundle/py2hwsw/blob/main/py2hwsw/scripts/iob_core.py}{View Source}


\subsection{Configuration class: iob\_conf.py}
\label{sec:iob_conf}
% SPDX-FileCopyrightText: 2025 IObundle
%
% SPDX-License-Identifier: MIT

%
% Main classes
%

The \href{https://github.com/IObundle/py2hwsw/blob/main/py2hwsw/scripts/iob_conf.py}{iob\_conf} class is used to represent a configuration option of the core.
This class contains a set of attributes, each preceded by a comment describing the purpose of the attribute.

% Show iob_conf attributes
% py2_macro: class_attributes iob_conf from iob_conf.py
\href{https://github.com/IObundle/py2hwsw/blob/main/py2hwsw/scripts/iob_conf.py}{View Source}


The \href{https://github.com/IObundle/py2hwsw/blob/main/py2hwsw/scripts/iob_conf.py}{iob\_conf\_group} class is used to represent a group of configuration options.
This class contains a set of attributes, each preceded by a comment describing the purpose of the attribute.

% Show iob_conf_group attributes
% py2_macro: class_attributes iob_conf_group from iob_conf.py
\href{https://github.com/IObundle/py2hwsw/blob/main/py2hwsw/scripts/iob_conf.py}{View Source}

%
% Generator methods
%

The py2hwsw tool uses methods from the \href{https://github.com/IObundle/py2hwsw/blob/main/py2hwsw/scripts/config_gen.py}{config\_gen.py} script to generate the `*\_conf.vh` file, which contains all the Verilog macros that must be held for every design instance of the core.

Each generated Verilog macro is based on the attributes from the corresponding instance of the `iob\_conf` class.

% Show group generation of conf_vh
% py2_macro: listing conf_vh from config_gen.py
\href{https://github.com/IObundle/py2hwsw/blob/main/py2hwsw/scripts/config_gen.py}{View Source}


The py2hwsw tool uses methods from the \href{https://github.com/IObundle/py2hwsw/blob/main/py2hwsw/scripts/param_gen.py}{param\_gen.py} script to generate the Verilog parameters code that is automatically inserted in the core's Verilog module and instances.

Each generated Verilog parameter is based on the attributes from the corresponding instance of the `iob\_conf` class.

% Show verilog param generation
% py2_macro: listing generate_params from param_gen.py
\href{https://github.com/IObundle/py2hwsw/blob/main/py2hwsw/scripts/param_gen.py}{View Source}


\subsection{Signal class: iob\_signal.py}
\label{sec:iob_signal}
% SPDX-FileCopyrightText: 2025 IObundle
%
% SPDX-License-Identifier: MIT

%
% Main classes
%

The \href{https://github.com/IObundle/py2hwsw/blob/main/py2hwsw/scripts/iob_wire.py}{iob\_wire} class is used to represent a wire for a hardware bus or port.
This class contains a set of attributes, each preceded by a comment describing the purpose of the attribute.

% Show attributes of iob_wire
% py2_macro: class_attributes iob_wire from iob_wire.py

%
% Generator methods
%

The py2hwsw tool uses the `get\_verilog\_bus`/`get\_verilog\_port` methods from the `iob\_wire` class to generate the Verilog code for the hardware bus/port based on the attributes from the corresponding instance of the `iob\_wire` class.

% Show functions that generate verilog for bus and port
% py2_macro: listing get_verilog_bus from iob_wire.py
% py2_macro: listing get_verilog_port from iob_wire.py


\subsection{Wire class: iob\_wire.py}
\label{sec:iob_wire}
% SPDX-FileCopyrightText: 2025 IObundle
%
% SPDX-License-Identifier: MIT

%
% Main classes
%

The \href{https://github.com/IObundle/py2hwsw/blob/main/py2hwsw/scripts/iob_wire.py}{iob\_wire} class is used to represent a group of hardware wires (signals) used to interconnect components automatically generated.
This class contains a set of attributes, each preceded by a comment describing the purpose of the attribute.

\lstinputlisting[firstline=21,lastline=35,language=Python]{iob_wire.py}
\href{https://github.com/IObundle/py2hwsw/blob/main/py2hwsw/scripts/iob_wire.py}{View Source}

The `signals` attribute stores a list of signal objects, represented by the `iob\_signal` class (Section~\ref{sec:iob_signal}).

%
% Generator methods
%

The py2hwsw tool uses the `generate\_wires` method from the `wire\_gen.py` script to generate the Verilog code for the wire based on the attributes from the corresponding instance of the `iob\_wire` class.

\lstinputlisting[firstline=20,lastline=37,language=Python]{wire_gen.py}
\href{https://github.com/IObundle/py2hwsw/blob/main/py2hwsw/scripts/wire_gen.py}{View Source}


\subsection{Port class: iob\_port.py}
\label{sec:iob_port}
% SPDX-FileCopyrightText: 2025 IObundle
%
% SPDX-License-Identifier: MIT

%
% Main classes
%

The \href{https://github.com/IObundle/py2hwsw/blob/main/py2hwsw/scripts/iob_port.py}{iob\_port} class is used to represent an interface for the core.
An interface is a group of hardware ports (signals) that may be generic or follow a standard.
Due to the similarities between a port and a wire, this class inherits the attributes from the `iob\_wire`~\ref{iob_wire} class.
Besides the inherited attributes, the class contains a set of new port specific attributes, each preceded by a comment describing the purpose of the attribute.

\lstinputlisting[firstline=19,lastline=29,language=Python]{iob_port.py}
\href{https://github.com/IObundle/py2hwsw/blob/main/py2hwsw/scripts/iob_port.py}{View Source}

Similar to the `iob\_wire` class, the `signals` attribute stores a list of signal objects, represented by the `iob\_signal` class~\ref{iob_signal}.

%
% Generator methods
%

The py2hwsw tool uses the `generate\_ports` method from the `io\_gen.py` script to generate the Verilog code for the port based on the attributes from the corresponding instance of the `iob\_port` class.

\lstinputlisting[firstline=28,lastline=53,language=Python]{io_gen.py}
\href{https://github.com/IObundle/py2hwsw/blob/main/py2hwsw/scripts/io_gen.py}{View Source}


\subsection{Special cases}
\label{sec:special_cases}
% SPDX-FileCopyrightText: 2024 IObundle
%
% SPDX-License-Identifier: MIT

Most of the cores provived by the py2hwsw's library are built using the standard interfaces mentioned in section~\ref{sec:py2_standard_interfaces}.

However, there are some cores that due to limitations of the standard interfaces, rely instead on internal py2hwsw methods for extra features.
The following list describes the cores don't rely solely on the standard interfaces.

\begin{itemize}
\item \textbf{iob\_system}: This core uses the `is\_system` attribute to enable an internal py2hwsw method that automatically fixes the address widths of the cbus interfaces of the system's peripherals.
\item \textbf{iob\_csrs}: The py2hwsw tool contains an internal method to automatically search for the "iob\_csrs" subblock and insert a "<prefix>\_cbus\_s" port on the instantiator core of this subblock. It then connects this newly created "<prefix>\_cbus\_s" port of the instantiator core to the iob\_csrs "control\_if\_s" port. The '<prefix>' is replaced by instance name of iob\_csrs subblock.
\end{itemize}






% \subsection{Short Notation}
% \label{sec:short_notation}
% \input{short_notation}

% TODO: Subsections below
% %
% % How To Use
% %
% \ifdefined\SECTIONCLEARPAGE
% \clearpage
% \fi
% \section{How To Use}
% \label{sec:usage}
% 
% \subsection{Setup}
% \label{sec:setup}
% % SPDX-FileCopyrightText: 2025 IObundle
%
% SPDX-License-Identifier: MIT

To set up a core with Py2HWSW, you'll need to have Nix installed on your system. You can download and install Nix from the official Nix website. Once Nix is installed, you can clone the Py2HWSW repository using the command \texttt{git clone --recursive git@github.com:IObundle/py2hwsw.git}.

Next, navigate to the Py2HWSW directory and run the command \texttt{nix-shell} to enter the Nix-shell environment. This will ensure that all dependencies required by Py2HWSW are installed and available.

To set up a core, you can use the command \texttt{nix-shell --run "py2hwsw $(CORE) setup --build_dir ’$(BUILD_DIR)’ --py_params ’param1=param1_val:param2=param2_val’"}. This command will generate the necessary files and directories for your core in the specified build directory.

You can customize the setup process by passing additional options to the \texttt{py2hwsw} command. For example, you can disable format and linting checks by adding the options \texttt{--no_verilog_lint} and \texttt{--no_verilog_format} to the command.

Here's an example of a setup directory structure:
\begin{verbatim}
mycore/
  mycore.py
  hardware/
    src/
      mycore.v
\end{verbatim}
In this example, the \texttt{mycore.py} file contains the core description, and the \texttt{hardware/src} directory contains the Verilog source files for the core.

In some cases, the Verilog source file (\texttt{mycore.v}) may not be necessary, as the \texttt{mycore.py} file can describe the entire core, including its ports, wires, components, and even custom Verilog code. This allows for a high degree of flexibility and customization, as users can define their core's architecture and behavior entirely within the Python description file.

The Python description is particularly useful because it enables the creation of higher-level abstractions, making it easier to design and work with complex hardware components. Additionally, the use of Python parameters allows for dynamic modification of cores, enabling users to easily customize and adapt their designs to different use cases and applications.

To set up this core, you would run the command \texttt{nix-shell --run "py2hwsw mycore setup --build_dir ’./build’ --py_params ’param1=param1_val:param2=param2_val’"}. This would generate the necessary files and directories for the core in the \texttt{./build} directory.

Note that you can customize the setup process to fit your specific needs by modifying the core description, Verilog source files, and setup command options.


% 
% \subsection{Simulation}
% \label{sec:tbbd}
% % SPDX-FileCopyrightText: 2025 IObundle
%
% SPDX-License-Identifier: MIT

To simulate a core using Py2HWSW, you can use the \texttt{make sim-run} command inside the generated build directory. This command will run the simulation using the default simulator (Icarus Verilog).
You can specify the simulator to be used using the \texttt{SIMULATOR} variable.

For example, to simulate the core using Verilator, you can run the command \texttt{make sim-run SIMULATOR=verilator} inside the core's build directory. This will compile the testbench and run the simulation, displaying the output on the console.

Py2HWSW also provides a universal Verilator testbench that can be used to simulate IP cores. The testbench behaves like a processor reading and writing to the core's control and status registers (CSRs), allowing for easy testing and verification of the core's functionality.

You can customize the simulation process by modifying the testbench and simulation parameters, such as the simulation time, input stimuli, and output signals to be monitored. Additionally, you can use other simulators, such as VCS or QuestaSim, by specifying the corresponding simulator variable.

Some cores in the Py2HWSW library also include a tester that can be used to verify their functionality. Examples of such cores include \texttt{iob_aoi}, \texttt{iob_pulse_gen}, and \texttt{iob_system}. These testers can be run along with the core to test its behavior and ensure that it is working as expected.

To run the tester, simply navigate to the tester's build directory, usually located inside the core's build directory in a folder named \texttt{<core_name>_tester/}, and run the command \texttt{make sim-run}.
This will compile and run the tester, allowing you to verify the core's functionality and debug any issues that may arise.
By providing these testers, Py2HWSW makes it easier to develop and test complex hardware components, and ensures that the cores in the library are reliable and functional.


% 
% \ifdefined\ASICSYNTH
% \subsection{ASIC Synthesis}
% \label{sec:synth}
% % SPDX-FileCopyrightText: 2025 IObundle
%
% SPDX-License-Identifier: MIT

To synthesize a core for an Application-Specific Integrated Circuit (ASIC) using Py2HWSW, you can use the \texttt{make asic} command in the generated build directory. This command will run the ASIC synthesis tools, such as Synopsys Design Compiler or Cadence Genus, to generate a netlist for the specified ASIC process.

Before running the \texttt{make asic} command, you'll need to specify the ASIC synthesizer that you want to use. You can do this by setting the \texttt{SYNTHESIZER} environment variable or by passing it as a command-line argument.

For example, to synthesize the core with yosys, you can run the command \texttt{make asic SYNTHESIZER=yosys}. This will generate a netlist that can be used as input for further ASIC design and verification tools, such as place and route, static timing analysis, and physical verification.

Py2HWSW supports a range of ASIC processes and libraries, and provides a set of pre-configured process files that make it easy to get started with ASIC synthesis. By using Py2HWSW to synthesize your cores, you can take advantage of the high performance and low power consumption of ASICs, while minimizing the complexity and effort required to develop and deploy your designs.
The list of available synthesis tools can be found in the subdirectories of the Py2HWSW repository syn folder \url{https://github.com/IObundle/py2hwsw/tree/main/py2hwsw/hardware/syn}.

% \fi
% 
% \ifdefined\FPGACOMP
% \subsection{FPGA Compilation}
% \label{sec:fpga}
% % SPDX-FileCopyrightText: 2025 IObundle
%
% SPDX-License-Identifier: MIT

To compile a core for an FPGA using Py2HWSW, you can use the \texttt{make fpga} command in the generated build directory.
This command will run the FPGA synthesis and implementation tools, such as Vivado or Quartus, to generate a bitstream for the specified FPGA board.

Before running the \texttt{make fpga} command, you'll need to specify the FPGA board that you want to target. You can do this by setting the \texttt{BOARD} environment variable or by passing is as a command-line argument.

For example, to compile the core for a Xilinx Artix-7 Basys 3 FPGA, you can run the command \texttt{make fpga BOARD=iob_basys3}. This will generate a bitstream that can be loaded onto the FPGA using the appropriate programming tools.

Py2HWSW supports a range of FPGA boards and devices, and provides a set of pre-configured board files that make it easy to get started with FPGA development. By using Py2HWSW to compile and implement your cores, you can take advantage of the flexibility and performance of FPGAs, while minimizing the complexity and effort required to develop and deploy your designs.
The list of available FPGA boards can be found in the subdirectories of the Py2HWSW repository fpga folder \url{https://github.com/IObundle/py2hwsw/tree/main/py2hwsw/hardware/fpga}.

% \fi
% 
% %
% % Configuration
% %
% \ifdefined\SECTIONCLEARPAGE
% \clearpage
% \fi
% \section{Configuration}
% \label{sec:config}
% % SPDX-FileCopyrightText: 2025 IObundle, Lda
%
% SPDX-License-Identifier: MIT
%
% Py2HWSW Version 0.81 has generated this code (https://github.com/IObundle/py2hwsw).


The following tables describe the IP core configuration. The core may be configured using macros or parameters:

\begin{description}
    \item \textbf{'M'} Macro: a Verilog macro or \texttt{define} directive is used to include or exclude code segments, to create core configurations that are valid for all instances of the core.
    \item \textbf{'P'} Parameter: a Verilog parameter is passed to each instance of the core and defines the configuration of that particular instance.
\end{description}

\begin{xltabular}{\textwidth}{|l|c|c|c|c|X|}

  \hline
  \rowcolor{iob-green}
  {\bf Configuration} & {\bf Type} & {\bf Min} & {\bf Typical} & {\bf Max} & {\bf Description} \\ \hline \hline

  \input general_operation_confs_tab

  \caption{General operation group}
\end{xltabular}
\label{general_operation_confs_tab:is}

The macros not listed above are constants. They improve the code readability and
should not be changed by the user. These constants are listed below:
\input constants

% 
% %
% % End to End Examples
% %
% \ifdefined\SECTIONCLEARPAGE
% \clearpage
% \fi
% \section{End to End Examples}
% \label{sec:examples}
% % SPDX-FileCopyrightText: 2025 IObundle
%
% SPDX-License-Identifier: MIT

This section provides three end-to-end examples of using Py2HWSW to generate and verify digital hardware cores. The examples cover the iob\_aoi, iob\_pulse\_gen, and iob\_soc cores, showcasing the automation of Verilog module generation from attributes.

\subsubsection{iob\_aoi Example}

The iob\_aoi core is a simple example that combines an AND, OR, and invert logic gates. The core's attributes are defined in the iob\_aoi.py file, available at \url{https://github.com/IObundle/py2hwsw/tree/main/py2hwsw/lib/hardware/basic_tests/iob_aoi}. To generate the iob\_aoi core, follow these steps:

\begin{enumerate}
\item Create or modify the iob\_aoi.py file to set the attributes of the core, describing how it should be generated using the Py2HWSW standard core dictionary interface.
\item Optionally, add more files to the setup directory as needed, such as manual Verilog sources or templates, scripts, or software.
\item Call the Py2HWSW setup process using the command:
\begin{lstlisting}[language=bash]
nix-shell --run "py2hwsw iob_aoi setup --build_dir '$(BUILD_DIR)'"
\end{lstlisting}
\item The generated build directory contains all the Verilog sources, Makefile, and configurations to run the core in various flows (simulation, FPGA).
\item To run the core in simulation, call the command: \texttt{make sim-run} from the build directory.
\end{enumerate}

\subsubsection{iob\_pulse\_gen Example}

The iob\_pulse\_gen core is used to generate signal pulses with configurable start and duration. The core's attributes are defined in the iob\_pulse\_gen.py file, available at \url{https://github.com/IObundle/py2hwsw/blob/main/py2hwsw/lib/hardware/clocks_resets/iob_pulse_gen/iob_pulse_gen.py}. To generate the iob\_pulse\_gen core, follow these steps:

\begin{enumerate}
\item Create or modify the iob\_pulse\_gen.py file to set the attributes of the core, describing how it should be generated using the Py2HWSW standard core dictionary interface.
\item Optionally, add more files to the setup directory as needed, such as manual Verilog sources or templates, scripts, or software.
\item Call the Py2HWSW setup process using the command:
\begin{lstlisting}[language=bash]
nix-shell --run "py2hwsw iob_pulse_gen setup --build_dir '$(BUILD_DIR)'"
\end{lstlisting}
\item The generated build directory contains all the Verilog sources, Makefile, and configurations to run the core in various flows (simulation, FPGA).
\item To run the core in simulation, call the command: \texttt{make sim-run} from the build directory.
\end{enumerate}

\subsubsection{iob\_soc Example}

The iob\_soc core is a more complex example used to create a system on chip. The core's attributes are defined in the iob\_soc.py file, available at \url{https://github.com/IObundle/iob-soc}. The iob\_soc.py file supports high-level Python parameters that allow configuring main SoC components like the CPU, memories, and peripherals. To generate the iob\_soc core, follow these steps:

\begin{enumerate}
\item Create or modify the iob\_soc.py file to set the attributes of the core, describing how it should be generated using the Py2HWSW standard core dictionary interface.
\item Optionally, add more files to the setup directory as needed, such as manual Verilog sources or templates, scripts, or software.
\item Call the Py2HWSW setup process using the command:
\begin{lstlisting}[language=bash]
nix-shell --run "py2hwsw iob_soc setup --build_dir '$(BUILD_DIR)'"
\end{lstlisting}
\item The generated build directory contains all the Verilog sources, Makefile, and configurations to run the core in various flows (simulation, FPGA).
\item To run the core in simulation, call the command: \texttt{make sim-run} from the build directory.
\end{enumerate}

These examples demonstrate the automation of Verilog module generation from attributes using Py2HWSW, showcasing the flexibility and ease of use of the framework.

% 
% %
% % FAQ
% %
% \ifdefined\SECTIONCLEARPAGE
% \clearpage
% \fi
% \section{FAQ}
% \label{sec:faq}
% \input{faq}

\end{document}
