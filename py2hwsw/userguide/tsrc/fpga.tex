% SPDX-FileCopyrightText: 2025 IObundle
%
% SPDX-License-Identifier: MIT

To compile a core for an FPGA using Py2HWSW, you can use the \texttt{make fpga} command in the generated build directory.
This command will run the FPGA synthesis and implementation tools, such as Vivado or Quartus, to generate a bitstream for the specified FPGA board.

Before running the \texttt{make fpga} command, you'll need to specify the FPGA board that you want to target. You can do this by setting the \texttt{BOARD} environment variable or by passing is as a command-line argument.

For example, to compile the core for a Xilinx Artix-7 Basys 3 FPGA, you can run the command \texttt{make fpga BOARD=iob_basys3}. This will generate a bitstream that can be loaded onto the FPGA using the appropriate programming tools.

Py2HWSW supports a range of FPGA boards and devices, and provides a set of pre-configured board files that make it easy to get started with FPGA development. By using Py2HWSW to compile and implement your cores, you can take advantage of the flexibility and performance of FPGAs, while minimizing the complexity and effort required to develop and deploy your designs.
The list of available FPGA boards can be found in the subdirectories of the Py2HWSW repository fpga folder \url{https://github.com/IObundle/py2hwsw/tree/main/py2hwsw/hardware/fpga}.
