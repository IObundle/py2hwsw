% SPDX-FileCopyrightText: 2025 IObundle
%
% SPDX-License-Identifier: MIT

Py2hwsw addresses several key challenges in the hardware design process:
\begin{itemize}
    \item \textbf{Complexity of Verilog Coding}: Writing Verilog code can be intricate and error-prone, especially for those who may not be deeply familiar with hardware description languages. Py2hwsw simplifies this by allowing designers to specify their hardware requirements using high-level Python or JSON dictionaries, reducing the need for extensive Verilog knowledge.

    \item \textbf{Integration of Existing Designs}: Many projects involve legacy Verilog cores that need to be integrated with new designs. Py2hwsw facilitates this integration, enabling users to leverage existing components while still benefiting from the tool's advanced features.

    \item \textbf{Configuration Challenges}: Customizing hardware components often requires deep dives into low-level code. Py2hwsw allows for high-level configuration through IOb parameters, making it easier for designers to adjust their designs without getting bogged down in the details of Verilog.

    \item \textbf{Resource Generation}: The process of preparing scripts and Makefiles for various deployment environments can be tedious and time-consuming. Py2hwsw automates this process, providing users with the necessary resources to run their designs on different FPGAs, simulators, and synthesis tools.

    \item \textbf{Code Readability and Maintenance}: Maintaining and debugging hardware designs can be challenging, especially when the code is not well-documented. Py2hwsw generates legible Verilog code with comments, enhancing readability and making it easier for teams to collaborate and maintain their designs over time.
\end{itemize}

In summary, Py2hwsw streamlines the hardware design workflow, making it more accessible, efficient, and manageable for engineers and designers.
