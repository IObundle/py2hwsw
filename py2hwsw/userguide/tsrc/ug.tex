% SPDX-FileCopyrightText: 2025 IObundle
%
% SPDX-License-Identifier: MIT

\documentclass{ug}

%
% Packages and configuration
%

\usepackage{xltabular}

\usepackage{hyperref}
\hypersetup{bookmarksnumbered=true}
% Set up hyperlink colors
\hypersetup{
    colorlinks=true, % false: boxed links; true: colored links
    linkcolor=blue, % color of internal links
    citecolor=blue, % color of citations
    urlcolor=blue   % color of external links
}


\usepackage{listings}
\usepackage{xcolor}
% Define colors
\definecolor{keywordcolor}{RGB}{0, 0, 255} % Blue for keywords
\definecolor{commentcolor}{RGB}{0, 128, 0} % Green for comments
\definecolor{stringcolor}{RGB}{255, 0, 0} % Red for strings
\definecolor{backgroundcolor}{RGB}{245, 245, 245} % Light gray background

% Set up listings
\lstdefinestyle{Python}{
    backgroundcolor=\color{backgroundcolor},
    basicstyle=\ttfamily,
    keywordstyle=\color{keywordcolor},
    commentstyle=\color{commentcolor},
    stringstyle=\color{stringcolor},
    numbers=left,
    numberstyle=\tiny,
    stepnumber=1,
    numbersep=5pt,
    showstringspaces=false,
    tabsize=4,
    breaklines=true
}
% Set the default style for all lstlisting environments
\lstset{style=Python}

% Support unicode chars for drawing directory trees (├ ─ └)
\usepackage{pmboxdraw}

%
% Document content
%

% SPDX-FileCopyrightText: 2025 IObundle
%
% SPDX-License-Identifier: MIT

%replace ipcore-name by the name of your ip core (e.g. IOb-Cache) and description by a brief description (e.g. a Configurable Cache)

\title{%
\Huge {\bf Py2HWSW} \\
 \vspace*{3cm}
\LARGE A Python Framework for HW/SW Co-design
}

\header{Py2HWSW, A Python Framework for HW/SW Co-design}

\date{\today}
\category{User Guide, \input{\NAME_version.tex}, Build \input{shortHash.tex}}

\input{color}

\begin{document}

\maketitle
\pagenumbering{gobble}

\vspace*{\fill}
User Guide, \input{\NAME_version.tex}, Build \input{shortHash.tex}
\hspace*{\fill} \includegraphics[keepaspectratio,scale=.7]{Logo.png}

\cleardoublepage
\pagenumbering{roman}
\setcounter{page}{1}
\input{revhist}
\cleardoublepage
\tableofcontents
\clearpage
\listoftables
\clearpage
\listoffigures
\cleardoublepage
\pagenumbering{arabic}
\setcounter{page}{1}

%
% Introduction
%
\section{Introduction}
\label{sec:intro}
% SPDX-FileCopyrightText: 2025 IObundle
%
% SPDX-License-Identifier: MIT

Open-source Python framework for managing files, automating
project flows of embedded hardware/software codesign projects, and partially generating Verilog
hardware components. The framework simplifies the project structure, addresses challenges in
Hardware Design Languages like Verilog and VHDL, and automates emulation, simulation,
FPGA, and ASIC flows. The proposed Verilog generator offers flexibility, user control, and ease
of use, producing human-readable code compatible across FPGAs and ASICs.

%\input{symb}

\subsection{What Is Py2HWSW?}
\label{sec:whatispy2}
% SPDX-FileCopyrightText: 2025 IObundle
%
% SPDX-License-Identifier: MIT

In the rapidly evolving landscape of hardware design, the need for efficient and flexible tools is paramount. Enter py2hwsw, a powerful tool designed to streamline the process of generating Verilog cores from high-level descriptions provided in Python or JSON dictionaries. With py2hwsw, engineers can easily translate their design specifications into functional hardware components, significantly reducing development time and complexity.


\subsection{What Is Py2HWSW For?}
\label{sec:purpose}
% SPDX-FileCopyrightText: 2025 IObundle
%
% SPDX-License-Identifier: MIT

Py2HWSW is designed to do the following:
\begin{itemize}
    \item \textbf{Core Generation}: Generates Verilog cores from descriptions in Python or JSON dictionaries.

    \item \textbf{Framework Compatibility}: Integrates seamlessly with existing Verilog cores and frameworks.

    \item \textbf{High-Level Configuration}: Allows configuration of cores via high-level IOb parameters.

    \item \textbf{Automated Resources}: Produces scripts and Makefiles for deployment in various FPGAs, simulators, and synthesis tools, along with documentation.

    \item \textbf{Readable Code}: Generates legible Verilog code with comments for better understanding and maintenance.

\end{itemize}


\subsection{What Problem Does Py2HWSW Solve?}
\label{sec:problem}
% SPDX-FileCopyrightText: 2025 IObundle
%
% SPDX-License-Identifier: MIT

Py2hwsw addresses several key challenges in the hardware design process:
\begin{itemize}
    \item \textbf{Complexity of Verilog Coding}: Writing Verilog code can be intricate and error-prone, especially for those who may not be deeply familiar with hardware description languages. Py2hwsw simplifies this by allowing designers to specify their hardware requirements using high-level Python or JSON dictionaries, reducing the need for extensive Verilog knowledge.

    \item \textbf{Integration of Existing Designs}: Many projects involve legacy Verilog cores that need to be integrated with new designs. Py2hwsw facilitates this integration, enabling users to leverage existing components while still benefiting from the tool's advanced features.

    \item \textbf{Configuration Challenges}: Customizing hardware components often requires deep dives into low-level code. Py2hwsw allows for high-level configuration through IOb parameters, making it easier for designers to adjust their designs without getting bogged down in the details of Verilog.

    \item \textbf{Resource Generation}: The process of preparing scripts and Makefiles for various deployment environments can be tedious and time-consuming. Py2hwsw automates this process, providing users with the necessary resources to run their designs on different FPGAs, simulators, and synthesis tools.

    \item \textbf{Code Readability and Maintenance}: Maintaining and debugging hardware designs can be challenging, especially when the code is not well-documented. Py2hwsw generates legible Verilog code with comments, enhancing readability and making it easier for teams to collaborate and maintain their designs over time.
\end{itemize}

In summary, Py2hwsw streamlines the hardware design workflow, making it more accessible, efficient, and manageable for engineers and designers.


\subsection{What Design Principles Underlie Py2HWSW?}
\label{sec:principles}
% SPDX-FileCopyrightText: 2025 IObundle
%
% SPDX-License-Identifier: MIT

Py2HWSW works by:
\begin{itemize}
    \item \textbf{Standard Py2HWSW syntax}: Use a standard Py2HWSW
      syntax to describe each core.
    \item \textbf{Support Python dictionaries and JSON files}: Supports Python
      dictionaries to generate dynamic cores based on IOb parameters. And
      supports JSON files to describe fixed cores and for compatibily with cores
      generated by external tools.
    \item \textbf{Support custom verilog snippets}: Each core may include custom
      verilog snippets for any edge-case which cannot be described using Py2HWSW
      syntax.
    \item \textbf{Internal Object-Oriented structure}: Py2HWSW converts core
      descriptions into its internal object-oriented system, creating high-level
      abstractions of Verilog building blocks.
\end{itemize}


\subsection{How Does Py2HWSW Accomplish Its Goals?}
\label{sec:how}
% SPDX-FileCopyrightText: 2025 IObundle
%
% SPDX-License-Identifier: MIT

\begin{itemize}
    \item \textbf{Two-Step Development Process}: The core development is divided
    into two distinct phases: the \textbf{setup} phase and the \textbf{build}
    phase. During the setup phase, Verilog source files are generated based on
    high-level descriptions provided in Python or JSON format. The build phase then
    utilizes these Verilog sources to produce the necessary FPGA bitstreams,
    netlists, and other deployment files.

    \item \textbf{Independent Setup Folders}: Each core is organized within its own
    independent setup folder, containing high-level description files and, if
    needed, low-level files as well.

    \item \textbf{Core Description Input}: The core's specifications are provided
    to Py2hwsw in the form of JSON or a Python dictionary, utilizing standard
    Py2hwsw attributes.

    \item \textbf{Flexible Attribute Handling}: When generating the cores
    dictionary via a Python script, users can include a set of standard Py2hwsw
    attributes alongside their own custom-defined attributes.
\end{itemize}


%
% Getting Started
%
\ifdefined\SECTIONCLEARPAGE
\clearpage
\fi
\section{Getting Started}
\label{sec:gs}

\subsection{Setup Directory}
\label{sec:setup_dir}
% SPDX-FileCopyrightText: 2025 IObundle
%
% SPDX-License-Identifier: MIT

The setup directory of a core may have the following structure:

\begin{verbatim}
.
├── core_name.py
├── core_name.json
├── document
│   ├── doc_build.mk
│   ├── figures
│   └── tsrc
├── hardware
│   ├── src
│   ├── fpga
│   │   ├── fpga_build.mk
│   │   ├── src
│   │   ├── quartus
│   │   └── vivado
│   ├── modules
│   ├── simulation
│   │   ├── sim_build.mk
│   │   └── src
│   └── syn
│       ├── src
│       └── genus
├── software
│   ├── sw_build.mk
│   └── src
├── scripts
├── submodules
├── Makefile
├── README.md
├── LICENSE
├── CITATION.cff
└── default.nix
\end{verbatim}

Only the \texttt{core\_name.py} or \texttt{core\_name.json} file is needed to
pass the core's description to Py2HWSW.  The remaining directories and files are
optional.

If the \texttt{document}, \texttt{hardware}, and \texttt{software} directories
exist, they will be copied to the \texttt{build} directory, overriding any files
already present there, such as standard ones or files from other cores.

The \texttt{*\_build.mk} files allow the user to include core specific Makefile
targets and variables from the build process.  These will be copied to the
\texttt{build} directory and included in the standard build process Makefiles.

The \texttt{src} directories contain manually written Verilog/C/TeX sources for
the core, should they be needed.

The following directories and files do not follow a mandatory structure, but are
typically used for the following purposes:

The \texttt{hardware/modules} and \texttt{submodules} directories typically
contain setup directories of other cores.

The \texttt{scripts} directory contains scripts specific to the core, and may be
called by the user or from the \texttt{core\_name.py} script.

A simple example of a core's setup directory is available for the
\href{https://github.com/IObundle/py2hwsw/tree/main/py2hwsw/lib/hardware/basic_tests/iob_and}{iob\_and}
core.

A more complex example of a core's setup directory is available for the
\href{https://github.com/IObundle/iob-soc}{iob\_soc} core.


\subsection{Create An AND Gate Core: iob\_and}
\label{sec:iob_and}
% SPDX-FileCopyrightText: 2025 IObundle
%
% SPDX-License-Identifier: MIT

The simplest core description for Py2HWSW is as follows:

% py2_macro: file iob_and.py

A set of basic cores to showcase the various Py2HWSW features can be found in
the \href{https://github.com/IObundle/py2hwsw/tree/main/py2hwsw/lib/hardware/basic_tests}{basic\_tests}
directory.


\subsection{Setup And Build}
\label{sec:setup_build}
% SPDX-FileCopyrightText: 2025 IObundle
%
% SPDX-License-Identifier: MIT

To checkout the source and setup the
example \href{https://github.com/IObundle/py2hwsw/tree/main/py2hwsw/lib/hardware/basic_tests/iob_and}{iob\_and}
core:

\begin{lstlisting}[language=bash]
$ git clone --recursive git@github.com:IObundle/py2hwsw.git
$ cd py2hwsw/
$ nix-shell py2hwsw/lib/ # Optional step to install environment with necessary dependencies
$ py2hwsw iob_and setup --no_verilog_lint
\end{lstlisting}

To do a clean setup:

\begin{lstlisting}[language=bash]
$ py2hwsw iob_and clean
$ py2hwsw iob_and setup --no_verilog_lint
\end{lstlisting}

The setup process will generate a build directory containing the core's verilog sources and build files.
By default, the build directory is `../[core\_name]\_V[core\_version]`.

To build and run the core in simulation:
\begin{lstlisting}[language=bash]
$ make -C ../iob_and_V* sim-run
\end{lstlisting}


\subsection{Installation}
\label{sec:installation}
% SPDX-FileCopyrightText: 2025 IObundle
%
% SPDX-License-Identifier: MIT

Py2HWSW uses a Nix-shell environment to handle dependencies. The full list of dependencies is available as Nix packages in the \texttt{default.nix} file, which can be found at \url{https://github.com/IObundle/py2hwsw/blob/main/py2hwsw/lib/default.nix}.

The recommended way to install Py2HWSW is by using Nix-shell. Most Makefiles in IObundle projects call Nix-shell by default, so it is expected that a user will install Py2HWSW via Nix-shell. To do this, simply install Nix by following the instructions at \url{https://nixos.org/download.html#nix-install-linux}. Then, navigate to a directory that contains the Py2HWSW \texttt{default.nix} file and run \texttt{nix-shell}. Py2HWSW will self-install, and all dependencies will be installed automatically.

Alternatively, it is possible to install Py2HWSW manually by removing the Nix-shell commands from the Makefiles and installing the dependencies manually. After doing so, the user can call Py2HWSW by adding the \texttt{py2hwsw} file from the \texttt{bin/} folder to the \texttt{PATH} environment variable. The \texttt{py2hwsw} file can be found at \url{https://github.com/IObundle/py2hwsw/blob/main/bin/py2hwsw}.

Another option is to install Py2HWSW using pip with the following command:
\begin{lstlisting}[language=bash]
pip install -e path/to/py2hwsw_directory
\end{lstlisting}
However, please note that this method is not officially supported, and dependencies will still need to be handled manually or by using Nix.

Py2HWSW is primarily maintained and tested on Linux, but it should also work on macOS and Windows Subsystem for Linux (WSL) since Nix is supported on these platforms.


\subsection{Basic Usage}
\label{sec:basic_usage}
% SPDX-FileCopyrightText: 2025 IObundle
%
% SPDX-License-Identifier: MIT

To use Py2HWSW, you can run the following command:
\begin{lstlisting}[language=bash]
nix-shell --run "py2hwsw $(CORE) setup --build_dir '$(BUILD_DIR)' --iob_params 'param1=param1_val:param2=param2_val"
\end{lstlisting}
This command sets up a core using Py2HWSW, where (CORE) is the name of the core, (BUILD DIR) is the directory where the build files will be generated, and (param1=param1\_val:param2=param2\_val) are optional IOb parameters that can be used to customize the core.

The --build\_dir option allows you to specify the location of the generated build directory. If not specified, the build directory will be generated in the parent directory of where the Py2HWSW command is called.

You can also use the --help option to list all available options and a brief description of each:
\begin{lstlisting}[language=bash]
py2hwsw --help
\end{lstlisting}

To create a new core, you will need to create a setup directory with the same name as the core. This directory should contain at least one file with the same name as the core, either with a .py or .json extension, that describes the core using attributes of the core dictionary. The setup directory may also contain other files and folders following a standard hierarchy, which is described in more detail in other sections of this document.

For examples of simple cores, you can refer to the basic\_tests folder in the Py2HWSW library: \url{https://github.com/IObundle/py2hwsw/tree/main/py2hwsw/lib/hardware/basic_tests}. For creating System On Chips, you can use the iob-soc repository as a template: \url{https://github.com/IObundle/iob-soc/tree/main}.

Some key concepts to understand when using Py2HWSW include:

\begin{itemize}
  \item Setup directory: The folder that contains the core description and base files, templates, scripts, and sources.
  \item Build directory: The folder generated by the Py2HWSW setup process, which contains a standard file hierarchy and all the necessary makefiles to build and run the core on various simulators, FPGA, ASIC tools, and linters.
  \item Core: An IP core that contains Verilog sources, documentation, scripts, high-level attributes, and possibly software.
  \item Module: Sometimes used as an alternative to core, but it is recommended to use the term "core" instead. May also refer to Verilog modules and Python modules.
\end{itemize}

Py2HWSW can be used to create a wide variety of cores, from simple to complex. One of the main advantages of using Py2HWSW is that it generates readable Verilog code and all the necessary makefiles to run the core on various flows, making it a powerful tool for hardware design and development.


\subsection{Universal Testbench}
\label{sec:utb}
% SPDX-FileCopyrightText: 2025 IObundle
%
% SPDX-License-Identifier: MIT

Py2HWSW supports a \textit{Universal Test Bench}.

To use the \textit{Universal Test Bench}, the core needs to provide the following files:
\begin{itemize}
  \item iob\_v\_tb.vh
  \item iob\_uut.v
  \item iob\_core\_tb.c
  \item sw\_build.mk
\end{itemize}


Create the \textbf{iob\_v\_tb.vh} testbench header source and define the \textbf{IOB\_CSRS\_ADDR\_W} macro to specify the address width of the simulation wrapper's CSRs bus (the width must be large enough address all CSRs from all verification instruments).
For example, the iob\_uart core's simulation wrapper only uses one verification instrument (the iob\_uart core itself). Therefore, the testbench should define the \textbf{IOB\_CSRS\_ADDR\_W} macro to have the same width as the iob\_uart core's CSRs bus.
The iob\_uart core's CSRs header files are also included because we can obtain the CSRs bus width from the auto-generated macro \textit{IOB\_UART\_CSRS\_ADDR\_W}
% py2_macro: file iob_uart/hardware/simulation/src/iob_v_tb.vh


Create \textbf{iob\_uut.v} simulation wrapper and instantiate the verification instruments.
For example, the iob\_uart core is also used as a verification instrument to test itself. It is instantiated in the uart's iob\_uut.v file, and its RS232 ports are connected in loopback. The iob\_uut.v file is generated from the iob\_uart\_sim.py core's attributes (using the Py2HWSW attribute: "name": "uut").
% py2_macro: file iob_uart_sim.py


Create the \textbf{iob\_core\_tb.c} source to drive the verification instruments (instantiated in the simulation wrapper).
For example, the iob\_uart core's testbench drives this core, writing data to it, and reading back the data received from the loopback.
% py2_macro: file iob_uart/software/src/iob_core_tb.c

Create the \textbf{sw\_build.mk} makefile segment and add the `tb` target to the `UTARGETS` list. Adding this target will cause the testbench software to be built. This testbench software will run on the host machine in parallel to the simulation.
Also add the verification instrument's CSRs sources to the `CSRS` list.
Also update the `TB\_INCLUDES` list as needed.
% py2_macro: file iob_uart/software/sw_build.mk


%
% How It Works
%
\ifdefined\SECTIONCLEARPAGE
\clearpage
\fi
\section{How It Works}
\label{sec:how_it_works}

This section gives a detailed description of the Py2HWSW framework.

\subsection{Overview}
\label{sec:py2_overview}
% SPDX-FileCopyrightText: 2025 IObundle
%
% SPDX-License-Identifier: MIT

The Py2HWSW framework is organized into a repository with several key folders and scripts. The repository contains the main Py2HWSW module, as well as a library of cores and peripherals. The framework uses a combination of Python scripts and Makefiles to automate the generation of build directories for hardware components.

The setup process in Py2HWSW begins with the user providing a description of the core, which can be in the form of a Python script or a JSON file, in a setup directory. This description is then used to trigger the setup process, which involves gathering all dependency cores and generating the necessary Verilog code. The setup process creates a build directory, where all the generated Verilog modules are stored, correctly connected and structured based on the user's description. The build directory is independent of Py2HWSW and can be used on any machine with the necessary toolchain.

The build process is a separate step that takes the generated build directory as input and uses the Makefiles to run the toolchain for a specific flow, such as simulation or FPGA synthesis. For example, in the case of FPGA synthesis, the build process takes the generated Verilog sources as input, generates a bitstream, uploads it to the FPGA, and runs the design. In the case of simulation, the build process takes the Verilog sources and generates a simulator executable (for Verilator) or runs the simulation directly. The build process can be run on any machine with the necessary toolchain, without requiring Py2HWSW to be installed.

The main launch script, py2hwsw.py, serves as the entry point for the framework, and is responsible for orchestrating the setup process. The script takes care of setting up the build environment, generating Verilog code, and creating the build directory. Once the build directory is generated, the user can run the build process independently of Py2HWSW, using the Makefiles to automate the simulation, synthesis, and compilation of the hardware components.


\subsection{Technical Details}
\label{sec:py2_technical_details}
% SPDX-FileCopyrightText: 2025 IObundle
%
% SPDX-License-Identifier: MIT


From the user's perspective, interacting with Py2HWSW is straightforward and intuitive. Users describe cores using dictionaries, lists, and strings, which are then converted internally into object representations of the correct class. The main attributes of Py2HWSW, such as ports, buses, and configuration, each have their own class, organizing the properties of each attribute. These attributes are described by a dictionary, where each key is a property, and are converted to the corresponding property of the class for the internal object representation when calling the Py2HWSW process.

As described in the "Standard Interfaces" section, users only need to interact with Py2HWSW using standard interfaces based on dictionaries, lists, and strings. Internally, Py2HWSW converts these inputs into object representations, but these are usually only modified by developers. A typical user does not need to understand the inner workings of Py2HWSW, making it easy to use and focus on designing and developing hardware components.

The iob\_core.py class is the central component of Py2HWSW, aggregating all the properties of an IP core. Its constructor is responsible for the setup process of the core, which involves converting and initializing the attributes of the core, setting up submodules (each one a new iob\_core instance), setting up superblocks (only if the current core is the top module or another superblock), and generating the sources for the current core in the build directory. If the current core is the top module, the setup process terminates by formatting and linting the code, as well as cleaning up temporary files from the build directory.

In terms of dependencies, Py2HWSW itself has a minimal set of requirements, including Python and certain Python libraries, as well as optional dependencies such as formatters and linters like Black, Verible, and Verilator. These formatters can be skipped if the user chooses not to use formatting and linting during setup. The generated build directory, on the other hand, may have additional dependencies specific to the build process, such as Makefiles, Verilog compilers and simulators. However, these dependencies are independent of Py2HWSW and are only required for the build process. Makefiles are not a required dependency of Py2HWSW itself, but can be useful for automating the setup process and integrating Py2HWSW into a larger project workflow. 


\subsection{Setup Flow Chart}
\label{sec:py2_flow_chart}

Figure~\ref{fig:py2_flow_chart} presents a high-level flow chart of the Py2HWSW setup procedure.

\begin{figure}[H]
  \centering {\includegraphics[width=\columnwidth]{py2_flow_chart.pdf}}
  \vspace{-0.7cm}
  \caption{High-Level Flow Chart of Py2HWSW Setup Procedure}
  \label{fig:py2_flow_chart}
\end{figure}

\subsection{Standard Interfaces}
\label{sec:py2_standard_interfaces}
% SPDX-FileCopyrightText: 2025 IObundle
%
% SPDX-License-Identifier: MIT

The Py2HWSW framework provides the following two standard interfaces:
\begin{enumerate}
  \item \textbf{IOb Parameters}: Core "setup" function receives information from Py2HWSW via a dictionary in its first argument, referred to as \textit{IOb Parameters}.
  \item \textbf{Core Dictionary}: Core "setup" function returns a core description dictionary to Py2HWSW, referred to as \textit{Core Dictionary}.
\end{enumerate}

The core's "setup" function is the python function defined by the user in the <core\_name>.py file.

If the core is described by a JSON file, then the \textit{IOb Parameters} interface is not available.
The JSON file gives a dictionary to Py2HWSW, similar to the python dictionary of the "setup" function.
This allows the user to use external tools to generate cores in JSON format.

%
% IOb parameters
%

The \textit{IOb Parameters} received by the core's "setup" function is a dictionary containing both parameters passed by its issuer and standard parameters passed by Py2HWSW.
Each key, value pair in the dictionary is a \textit{IOb Parameter}.
The value of the IOb parameter may be of any data type.

\begin{xltabular}{\textwidth}{|l|l|X|}

  \hline
  \rowcolor{iob-green}
  {\bf Name} & {\bf Data Type} & {\bf Description}  \\ \hline \hline

  \input py2hwsw_iob_params_tab

  \caption{Standard \textit{IOb Parameters} passed by Py2HWSW to every core's "setup" function.}
\end{xltabular}
\label{py2hwsw_iob_params_tab}

The standard IOb parameters passed by Py2HWSW are listed in Table~\ref{py2hwsw_iob_params_tab}.

The IOb parameters supported by each core is available in the respective core's user guide, as long as they have the \textit{IOb Parameters} attribute defined.
Instructions on how to build a core's user guide can be found in Section~\ref{sec:core_lib}. 


%
% Core dictionary
%

\begin{xltabular}{\textwidth}{|l|l|X|}
  \hline
  \rowcolor{iob-green}
  % TODO: The "Data Type" column should specify what the user should input, instead of the internal object used by py2hwsw.
  {\bf Name} & {\bf Data Type} & {\bf Description}  \\ \hline \hline
  \endhead

  \input py2hwsw_attributes_tab

  \caption{Table of supported Py2HWSW attributes in the \textbf{Core Dictionary}. The \textit{Data Type} column specifies the type of internal object that the Py2HWSW will convert the attribute's value to (usually the user inputs a string, list, or dictionary value and then py2 converts it to an internal object).}
  \label{py2hwsw_attributes_tab}
\end{xltabular}

The list of attributes supported by the Py2HWSW framework is given in Table~\ref{py2hwsw_attributes_tab}.
If a core provides a dictionary with keys not listed in Table~\ref{py2hwsw_attributes_tab}, then the Py2HWSW framework will raise an error.
Each key, value pair in the dictionary is a \textit{Core Attribute}.
The data type of the core attribute may be of any data type, but are usually a string, list, or dictionary.
If the data type is a string, it may also represent an object using Py2HWSW's \textit{Short Notation}.
%~\ref{sec:short_notation}


\subsection{Block hierarchy}
\label{sec:py2_block_hierarchy}

Figure~\ref{fig:py2_superblocks_subblocks} presents an example block hierarchy for a Py2HWSW project.
Superblocks are only used if they are superblocks of the project's top module or of one of its wrappers.

\begin{figure}[H]
  \centering {\includegraphics[width=\columnwidth]{superblocks_subblocks.pdf}}
  \vspace{-0.7cm}
  \caption{Block Hierarchy of a Py2HWSW Project}
  \label{fig:py2_superblocks_subblocks}
\end{figure}

\subsection{Main launch script: py2hwsw.py}
\label{sec:launch_script}
% SPDX-FileCopyrightText: 2025 IObundle
%
% SPDX-License-Identifier: MIT

The main launch script for the Py2HWSW progam is the `py2hwsw.py` script.

The following code snippet from that script processes the command line arguments and launches the program for the specified "target".

% py2_macro: start_line "iob_core.global_build_dir =" end_line "iob_core.deliver_core" from py2hwsw.py




\subsection{Simulate with Verilator}
\label{sec:verilator}
% SPDX-FileCopyrightText: 2025 IObundle
%
% SPDX-License-Identifier: MIT

With mandatory structured IOs, the testbench behaves like a processor reading
and writing to its CSR. A universal Verilator testbench has been developed for
an IP with a structured IOb native interface (bridges to standard AXI-Lite, APB
or Wishbone are supplied). The testbench is a C++ program provides hardware
reset and CSR read and write functions.

\subsubsection{IP core simulation}

The IP cores using this testbench must provide a C function called
{\tt iob\_core\_tb()}, the IP core’s specific test. They also must provide a C header
called {\tt iob\_vlt\_tb.h} that defines the Device Under Test (DUT) as a Verilator
type called {\tt dut\_t}. With knowledge of the DUT and its test, the universal
Verilator testbench will exercise any IP core. Interestingly, {\tt iob\_core\_tb()} also
runs, without modifications, on a RISC-V processor with the IP as a submodule,
for example, for FPGA testing or emulation.

The {\tt iob\_uart} core is used as an example, located in the {\tt
  py2hwsw/lib/peripherals/iob\_uartiob-uart} directory.

\begin{lstlisting}[language=bash]
  $ git clone --recursive git@github.com:IObundle/py2hwsw.git
  $ cd py2hwsw/lib
  $ make sim-run CORE=iob\_uart SIMULATOR=verilator
\end{lstlisting}

The {\tt make sim-run} command will run core setup, creating the build directory
at {\tt ../../../iob\_uart\_V0.1}. The Verilator simulator will be run in the
build directory. The testbench will be compiled and run, and the output will be
displayed on the console.

\subsubsection{Subsystem simulation}

To illustrate system test capabilities with the universal Verilator testbench,
the {\tt iob\_system} subsystem core is used as an example, located in the {\tt
  py2hwsw/lib/iob\_system} directory.

\begin{lstlisting}[language=bash]
  $ git clone --recursive git@github.com:IObundle/py2hwsw.git
  $ cd py2hwsw/lib
  $ make sim-run CORE=iob\_uart SIMULATOR=verilator
\end{lstlisting}

In this case the {\tt iob\_core\_tb()} function is running on the desktop,
emualting a system tester. The console output comes from teh system itself
runnig its embedded test, a more elaborated form of a hello world program.


\subsection{Deliver an IP core}
\label{sec:deliver}
% SPDX-FileCopyrightText: 2025 IObundle
%
% SPDX-License-Identifier: MIT

From the build directory, we select the essential files to create a tarball, all
containing a Makefile-driven environment for the user who, in this way, will not
need any ancillary tools beyond the standard EDA tools.

\begin{lstlisting}[language=bash]
  $ git clone --recursive git@github.com:IObundle/py2hwsw.git
  $ cd py2hwsw/
  $ nix-shell py2hwsw/lib/ # Optional step to install environment with necessary dependencies
  $ py2hwsw iob_uart setup --no_verilog_lint
  $ py2hwsw iob_uart deliver
\end{lstlisting}

The tarball wil be created in the \texttt{../iob\_uart\_V0.1} directory, which is
also the home of the default build directory.

%
% Py2HWSW classes
%
\section{Py2HWSW Classes}
\label{sec:py_classes}

\subsection{Main class for core representation: iob\_core.py}
\label{sec:iob_core}
% SPDX-FileCopyrightText: 2025 IObundle
%
% SPDX-License-Identifier: MIT

The \href{https://github.com/IObundle/py2hwsw/blob/main/py2hwsw/scripts/iob_core.py}{iob\_core} class is a central component of the Py2HWSW framework, responsible for representing and managing IP cores. The class provides a structured way to describe and generate IP cores, making it easier to create and integrate complex digital designs. At the heart of the iob\_core class is its constructor, which plays a crucial role in setting up and initializing the core.

The constructor of the iob\_core class is responsible for converting and initializing the attributes of the core, setting up subblocks, and generating the sources for the current core in the build directory. When the constructor is called, it distinguishes between two situations: when it is the top module, it creates a build directory for users with all the dependencies and project flows, and when it is a subblock, it provides all the information necessary for instantiation and integration. The constructor takes care of setting up the build environment, generating Verilog code, and creating the build directory, making it a key component of the Py2HWSW framework.

The iob\_core constructor also handles the setup process for subblocks and superblocks. When a subblock is encountered, the constructor is called recursively to set up the subblock's attributes and generate its sources. Similarly, when a superblock is encountered, the constructor sets up the superblock's attributes and generates its sources, ensuring that the entire hierarchy of modules and subblocks is properly initialized and configured, as detailed in Section~\ref{sec:py2_block_hierarchy}

Overall, the iob\_core class and its constructor provide a powerful and flexible way to represent and manage IP cores, making it a fundamental component of the Py2HWSW framework.

It inherits attributes from its parent classes \href{https://github.com/IObundle/py2hwsw/blob/main/py2hwsw/scripts/iob_module.py}{iob\_module} and \href{https://github.com/IObundle/py2hwsw/blob/main/py2hwsw/scripts/iob_instance.py}{iob\_instance}.

% get_core_obj function

The \href{https://github.com/IObundle/py2hwsw/blob/main/py2hwsw/scripts/iob_core.py#L858}{get\_core\_obj} function is used to generate an instance of a core based on a given core name and IOb parameters.
This method will search for the corresponding Python or JSON file of the core, and generate a python object based on info stored in that file, and info passed via IOb parameters.

% py2_macro: listing get_core_obj from iob_core.py


\subsection{Configuration class: iob\_conf.py}
\label{sec:iob_conf}
% SPDX-FileCopyrightText: 2025 IObundle
%
% SPDX-License-Identifier: MIT

%
% Main classes
%

The \href{https://github.com/IObundle/py2hwsw/blob/main/py2hwsw/scripts/iob_conf.py}{iob\_conf} class is used to represent a configuration option of the core.
This class contains a set of attributes, each preceded by a comment describing the purpose of the attribute.

% Show iob_conf attributes
% py2_macro: class_attributes iob_conf from iob_conf.py


%
% Generator methods
%

The py2hwsw tool uses methods from the \href{https://github.com/IObundle/py2hwsw/blob/main/py2hwsw/scripts/config_gen.py}{config\_gen.py} script to generate the `*\_conf.vh` file, which contains all the Verilog macros that must be held for every design instance of the core.

Each generated Verilog macro is based on the attributes from the corresponding instance of the `iob\_conf` class.

% Show group generation of conf_vh
% py2_macro: listing conf_vh from config_gen.py


The py2hwsw tool uses methods from the \href{https://github.com/IObundle/py2hwsw/blob/main/py2hwsw/scripts/param_gen.py}{param\_gen.py} script to generate the Verilog parameters code that is automatically inserted in the core's Verilog module and instances.

Each generated Verilog parameter is based on the attributes from the corresponding instance of the `iob\_conf` class.

% Show verilog param generation
% py2_macro: listing generate_params from param_gen.py


\subsection{Wire class: iob\_wire.py}
\label{sec:iob_wire}
% SPDX-FileCopyrightText: 2025 IObundle
%
% SPDX-License-Identifier: MIT

%
% Main classes
%

The \href{https://github.com/IObundle/py2hwsw/blob/main/py2hwsw/scripts/iob_bus.py}{iob\_bus} class is used to represent a group of hardware buses (signals) used to interconnect components automatically generated.
This class contains a set of attributes, each preceded by a comment describing the purpose of the attribute.

% Show attributes of iob_bus
% py2_macro: class_attributes iob_bus from iob_bus.py

The `signals` attribute stores a list of signal objects, represented by the `iob\_signal` class (Section~\ref{sec:iob_signal}).

%
% Generator methods
%

The py2hwsw tool uses the `generate\_buses` method from the `bus\_gen.py` script to generate the Verilog code for the bus based on the attributes from the corresponding instance of the `iob\_bus` class.

% Show functions that generate verilog buses
% py2_macro: listing generate_buses from bus_gen.py


\subsection{Bus class: iob\_bus.py}
\label{sec:iob_bus}
\input{iob_bus}

\subsection{Port class: iob\_port.py}
\label{sec:iob_port}
% SPDX-FileCopyrightText: 2025 IObundle
%
% SPDX-License-Identifier: MIT

%
% Main classes
%

The \href{https://github.com/IObundle/py2hwsw/blob/main/py2hwsw/scripts/iob_port.py}{iob\_port} class is used to represent an interface for the core.
An interface is a group of hardware ports (signals) that may be generic or follow a standard.
Due to the similarities between a port and a bus, this class inherits the attributes from the `iob\_bus` class (Section~\ref{sec:iob_bus}).
Besides the inherited attributes, the class contains a set of new port specific attributes, each preceded by a comment describing the purpose of the attribute.

% Show attributes of iob_port
% py2_macro: class_attributes iob_port from iob_port.py

Similar to the `iob\_bus` class, the `signals` attribute stores a list of signal objects, represented by the `iob\_signal` class  (Section~\ref{sec:iob_signal}).

%
% Generator methods
%

The py2hwsw tool uses the `generate\_ports` method from the `io\_gen.py` script to generate the Verilog code for the port based on the attributes from the corresponding instance of the `iob\_port` class.

% Show verilog generator function for ports
% py2_macro: listing generate_ports from io_gen.py


\subsection{Interface class: if\_gen.py}
\label{sec:if_gen}
% SPDX-FileCopyrightText: 2025 IObundle
%
% SPDX-License-Identifier: MIT

%
% Main classes
%

The Py2HWSW tool uses the \href{https://github.com/IObundle/py2hwsw/blob/main/py2hwsw/scripts/if_gen.py}{if\_gen.py} script to generate the signals of standard interfaces.

The list of standard interfaces currenlty supported by the if\_gen.py script are listed below.

% py2_macro: listing mem_if_details from if_gen.py
% py2_macro: listing if_details from if_gen.py

When a user specifies a standard interface for a port~\ref{sec:iob_port} or a bus~\ref{sec:iob_bus}, a new instance of the `interface` class is created to represent it.

% py2_macro: class_attributes interface from if_gen.py

This class stores the interface properties which will then be used by interface specific functions to generate the signals.
Each attribute of the interface class is preceded by a comment describing the purpose of the attribute.

Each interface supported by if\_gen.py contains its own `get\_\textless interface\textgreater\_ports` function, and returns a list of signals for the interface.

%
% AXI Stream
%
\clearpage
\large\textbf{AXI Stream}

The AXI Stream interface uses the `get\_axis\_ports` function of the \href{https://github.com/IObundle/py2hwsw/blob/main/py2hwsw/scripts/if_gen.py}{if\_gen.py} script.

% py2_macro: listing get_axis_ports from if_gen.py

It has the configurable width: DATA\_W

For example, to add an AXI Stream port to a core, add the following python dictionary to the core's ports list:
\begin{lstlisting}[language=python]
{
	"name": "example_port_m",
	"descr": "AXI Stream manager port",
	"signals": {
		"type": "axis",
		"DATA_W": 32,
	},
},
\end{lstlisting}


%
% AXI Lite
%
\clearpage
\large\textbf{AXI Lite}

The AXI Lite interface uses the `get\_axil\_ports` function of the \href{https://github.com/IObundle/py2hwsw/blob/main/py2hwsw/scripts/if_gen.py}{if\_gen.py} script.

% py2_macro: listing get_axil_ports from if_gen.py
% py2_macro: listing get_axil_read_ports from if_gen.py
% py2_macro: listing get_axil_write_ports from if_gen.py

It has the configurable widths: 
\begin{itemize}
  \item ADDR\_W
  \item DATA\_W
  \item PROT\_W
  \item RESP\_W
\end{itemize}

The AXI Lite interface also supports the the following 'params':
\begin{itemize}
  \item prot: Include "prot" signal
\end{itemize}

For example, to add an AXI Lite port to a core, add the following python dictionary to the core's ports list:
\begin{lstlisting}[language=python]
{
	"name": "example_port_m",
	"descr": "AXI Lite manager port",
	"signals": {
		"type": "axil",
		"ADDR_W": 32,
		"DATA_W": 32,
		"PROT_W": 3,
		"RESP_W": 2,
	},
},
\end{lstlisting}


%
% AXI
%
\clearpage
\large\textbf{AXI}

The AXI interface uses the `get\_axi\_ports` function of the \href{https://github.com/IObundle/py2hwsw/blob/main/py2hwsw/scripts/if_gen.py}{if\_gen.py} script.

% py2_macro: listing get_axi_ports from if_gen.py
% py2_macro: listing get_axi_read_ports from if_gen.py
% py2_macro: listing get_axi_write_ports from if_gen.py

The AXI interface extends configurable widths of the AXI Lite interface, with the following additions:
\begin{itemize}
  \item ID\_W
  \item SIZE\_W
  \item BURST\_W
  \item LOCK\_W
  \item CACHE\_W
  \item QOS\_W
  \item LEN\_W
\end{itemize}

The AXI interface supports the same 'params' as the AXI Lite interface.

For example, to add an AXI port to a core, add the following python dictionary to the core's ports list:
\begin{lstlisting}[language=python]
{
	"name": "example_port_m",
	"descr": "AXI manager port",
	"signals": {
		"type": "axi",
		"ADDR_W": 32,
		"DATA_W": 32,
		"PROT_W": 3,
		"RESP_W": 2,
		"ID_W": 4,
		"SIZE_W": 3,
		"BURST_W": 2,
		"LOCK_W": 2,
		"CACHE_W": 4,
		"QOS_W": 4,
		"LEN_W": 8,
	},
},
\end{lstlisting}


%
% APB
%
\clearpage
\large\textbf{APB}

The APB interface uses the `get\_apb\_ports` function of the \href{https://github.com/IObundle/py2hwsw/blob/main/py2hwsw/scripts/if_gen.py}{if\_gen.py} script.

% py2_macro: listing get_apb_ports from if_gen.py

The APB interface has the configurable widths:
\begin{itemize}
  \item ADDR\_W
  \item DATA\_W
\end{itemize}

For example, to add an APB port to a core, add the following python dictionary to the core's ports list:
\begin{lstlisting}[language=python]
{
	"name": "example_port_m",
	"descr": "APB manager port",
	"signals": {
		"type": "apb",
		"ADDR_W": 32,
		"DATA_W": 32,
	},
},
\end{lstlisting}


%
% AHB
%
\clearpage
\large\textbf{AHB}

The AHB interface uses the `get\_ahb\_ports` function of the \href{https://github.com/IObundle/py2hwsw/blob/main/py2hwsw/scripts/if_gen.py}{if\_gen.py} script.

% py2_macro: listing get_ahb_ports from if_gen.py

The AHB interface has the configurable widths:
\begin{itemize}
  \item ADDR\_W
  \item DATA\_W
  \item BURST\_W
  \item PROT\_W
  \item SIZE\_W
  \item TRANS\_W
\end{itemize}

For example, to add an AHB port to a core, add the following python dictionary to the core's ports list:
\begin{lstlisting}[language=python]
{
	"name": "example_port_m",
	"descr": "AHB manager port",
	"signals": {
		"type": "ahb",
		"ADDR_W": 32,
		"DATA_W": 32,
		"BURST_W": 3,
		"PROT_W": 4,
		"SIZE_W": 3,
		"TRANS_W": 2,
	},
},
\end{lstlisting}


%
% Wishbone
%
\clearpage
\large\textbf{Wishbone}

The Wishbone interface uses the `get\_wb\_ports` function of the \href{https://github.com/IObundle/py2hwsw/blob/main/py2hwsw/scripts/if_gen.py}{if\_gen.py} script.

% py2_macro: listing get_wb_ports from if_gen.py

The Wishbone interface has the configurable widths:
\begin{itemize}
  \item ADDR\_W
  \item DATA\_W
\end{itemize}

For example, to add an Wishbone port to a core, add the following python dictionary to the core's ports list:
\begin{lstlisting}[language=python]
{
	"name": "example_port_m",
	"descr": "Wishbone manager port",
	"signals": {
		"type": "wb",
		"ADDR_W": 32,
		"DATA_W": 32,
	},
},
\end{lstlisting}


%
% IOb
%
\clearpage
\large\textbf{IOb Native}

The IOb Native interface is an open source interface developed by IObundle.
It simplifies the connections between core components due to its reduced amount of signals when compared to other standard interfaces.
The Py2HWSW core library~\ref{sec:core_lib} provides core's to convert between the IOb and other standard interfaces.
A description of the IOb interface is available at:
\url{https://github.com/IObundle/py2hwsw/blob/main/py2hwsw/lib/hardware/buses/iob_tasks/README.md}

The IOb interface uses the `get\_iob\_ports` function of the \href{https://github.com/IObundle/py2hwsw/blob/main/py2hwsw/scripts/if_gen.py}{if\_gen.py} script.

% py2_macro: listing get_iob_ports from if_gen.py

The IOb interface has the configurable widths:
\begin{itemize}
  \item ADDR\_W
  \item DATA\_W
\end{itemize}

For example, to add an IOb port to a core, add the following python dictionary to the core's ports list:
\begin{lstlisting}[language=python]
{
	"name": "example_port_m",
	"descr": "IOb manager port",
	"signals": {
		"type": "iob",
		"ADDR_W": 32,
		"DATA_W": 32,
	},
},
\end{lstlisting}




\subsection{Special cases}
\label{sec:special_cases}
% SPDX-FileCopyrightText: 2025 IObundle
%
% SPDX-License-Identifier: MIT

Most of the cores provived by the py2hwsw's library are built using the standard interfaces mentioned in section~\ref{sec:py2_standard_interfaces}.

However, there are some cores that due to limitations of the standard interfaces, rely instead on internal py2hwsw methods for extra features.
The following list describes the cores don't rely solely on the standard interfaces.

\begin{itemize}
\item \textbf{iob\_system}: This core uses the `is\_system` attribute to enable an internal py2hwsw method that automatically fixes the address widths of the cbus interfaces of the system's peripherals.
\item \textbf{iob\_csrs}: The py2hwsw tool contains an internal method to automatically search for the "iob\_csrs" subblock and insert a "\textless prefix\textgreater \_cbus\_s" port on the issuer core of this subblock. It then connects this newly created "\textless prefix\textgreater \_cbus\_s" port of the issuer core to the iob\_csrs "control\_if\_s" port. The '\textless prefix\textgreater ' is replaced by instance name of iob\_csrs subblock.
\end{itemize}




\subsection{Core library}
\label{sec:core_lib}
% SPDX-FileCopyrightText: 2025 IObundle
%
% SPDX-License-Identifier: MIT

The Py2HWSW framework includes a \href{https://github.com/IObundle/py2hwsw/tree/main/py2hwsw/lib}{library of cores} ready for use.

\begin{xltabular}{\textwidth}{|l|l|X|}
  \hline
  \rowcolor{iob-green}
  {\bf Name} & {\bf Directory}  \\ \hline \hline
  \endhead

  \input py2hwsw_core_lib_tab

  \caption{Table of cores available in the library of the Py2HWSW framework. The \textit{Directory} column is the path to the core's setup directory, relative to the Py2HWSW lib directory \texttt{py2hwsw/lib/}.}
  \label{py2hwsw_core_lib_tab}
\end{xltabular}

Table~\ref{py2hwsw_core_lib_tab} lists the cores available in the Py2HWSW framework's core library.

Each core contains its own user guide, which can be built using the following commands:

\begin{lstlisting}
py2hwsw <core_name> setup
make -C ../<core_name>_V<core_version>/ doc-build
xdg-open ../<core_name>_V<core_version>/document/ug.pdf
\end{lstlisting}


%
% How To Use
%
\ifdefined\SECTIONCLEARPAGE
\clearpage
\fi
\section{How To Use}
\label{sec:usage}

\subsection{Setup}
\label{sec:setup}
% SPDX-FileCopyrightText: 2025 IObundle
%
% SPDX-License-Identifier: MIT

To set up a core with Py2HWSW, you'll need to have Nix installed on your system. You can download and install Nix from the official Nix website. Once Nix is installed, you can clone the Py2HWSW repository using the command \texttt{git clone --recursive git@github.com:IObundle/py2hwsw.git}.

Next, navigate to the Py2HWSW directory and run the command \texttt{nix-shell} to enter the Nix-shell environment. This will ensure that all dependencies required by Py2HWSW are installed and available.

To set up a core, you can use the command:
\begin{lstlisting}[language=bash]
nix-shell --run "py2hwsw $(CORE) setup --build_dir '$(BUILD_DIR)' --py_params 'param1=param1_val:param2=param2_val'"
\end{lstlisting}

This command will generate the necessary files and directories for your core in the specified build directory.

You can customize the setup process by passing additional options to the \texttt{py2hwsw} command. For example, you can disable format and linting checks by adding the options \texttt{--no\_verilog\_lint} and \texttt{--no\_verilog\_format} to the command.

Here's an example of a setup directory structure:
\begin{verbatim}
mycore/
  mycore.py
  hardware/
    src/
      mycore.v
\end{verbatim}
In this example, the \texttt{mycore.py} file contains the core description, and the \texttt{hardware/src} directory contains the Verilog source files for the core.

In some cases, the Verilog source file (\texttt{mycore.v}) may not be necessary, as the \texttt{mycore.py} file can describe the entire core, including its ports, buses, components, and even custom Verilog code. This allows for a high degree of flexibility and customization, as users can define their core's architecture and behavior entirely within the Python description file.

The Python description is particularly useful because it enables the creation of higher-level abstractions, making it easier to design and work with complex hardware components. Additionally, the use of Python parameters allows for dynamic modification of cores, enabling users to easily customize and adapt their designs to different use cases and applications.

To set up this core, you would run the command:
\begin{lstlisting}[language=bash]
nix-shell --run "py2hwsw mycore setup --build_dir './build' --py_params 'param1=param1_val:param2=param2_val'"
\end{lstlisting}

This would generate the necessary files and directories for the core in the \texttt{./build} directory.

Note that you can customize the setup process to fit your specific needs by modifying the core description, Verilog source files, and setup command options.



\subsection{Simulation}
\label{sec:sim}
% SPDX-FileCopyrightText: 2025 IObundle
%
% SPDX-License-Identifier: MIT

The provided testbench implements a self-loop, where the IOb-UART handshakes
({\tt cts} to {\tt rts}, and sends data to itself ({\tt txd} to {\tt rxd}). The
testbench drives the clock and reset wires, and emulates the CPU actions with
a simple control block.


\ifdefined\ASICSYNTH
\subsection{ASIC Synthesis}
\label{sec:synth}
% SPDX-FileCopyrightText: 2025 IObundle
%
% SPDX-License-Identifier: MIT

To synthesize a core for an Application-Specific Integrated Circuit (ASIC) using Py2HWSW, you can use the \texttt{make asic} command in the generated build directory. This command will run the ASIC synthesis tools, such as Synopsys Design Compiler or Cadence Genus, to generate a netlist for the specified ASIC process.

Before running the \texttt{make asic} command, you'll need to specify the ASIC synthesizer that you want to use. You can do this by setting the \texttt{SYNTHESIZER} environment variable or by passing it as a command-line argument.

For example, to synthesize the core with yosys, you can run the command \texttt{make asic SYNTHESIZER=yosys}. This will generate a netlist that can be used as input for further ASIC design and verification tools, such as place and route, static timing analysis, and physical verification.

Py2HWSW supports a range of ASIC processes and libraries, and provides a set of pre-configured process files that make it easy to get started with ASIC synthesis. By using Py2HWSW to synthesize your cores, you can take advantage of the high performance and low power consumption of ASICs, while minimizing the complexity and effort required to develop and deploy your designs.
The list of available synthesis tools can be found in the subdirectories of the Py2HWSW repository syn folder \url{https://github.com/IObundle/py2hwsw/tree/main/py2hwsw/hardware/syn}.

\fi

\ifdefined\FPGACOMP
\subsection{FPGA Compilation}
\label{sec:fpga}
% SPDX-FileCopyrightText: 2025 IObundle
%
% SPDX-License-Identifier: MIT

To compile a core for an FPGA using Py2HWSW, you can use the \texttt{make fpga} command in the generated build directory.
This command will run the FPGA synthesis and implementation tools, such as Vivado or Quartus, to generate a bitstream for the specified FPGA board.

Before running the \texttt{make fpga} command, you'll need to specify the FPGA board that you want to target. You can do this by setting the \texttt{BOARD} environment variable or by passing is as a command-line argument.

For example, to compile the core for a Xilinx Artix-7 Basys 3 FPGA, you can run the command \texttt{make fpga BOARD=iob_basys3}. This will generate a bitstream that can be loaded onto the FPGA using the appropriate programming tools.

Py2HWSW supports a range of FPGA boards and devices, and provides a set of pre-configured board files that make it easy to get started with FPGA development. By using Py2HWSW to compile and implement your cores, you can take advantage of the flexibility and performance of FPGAs, while minimizing the complexity and effort required to develop and deploy your designs.
The list of available FPGA boards can be found in the subdirectories of the Py2HWSW repository fpga folder \url{https://github.com/IObundle/py2hwsw/tree/main/py2hwsw/hardware/fpga}.

\fi

\ifdefined\SECTIONCLEARPAGE
\clearpage
\fi
\subsection{End to End Examples}
\label{sec:examples}
% SPDX-FileCopyrightText: 2025 IObundle
%
% SPDX-License-Identifier: MIT

This section provides three end-to-end examples of using Py2HWSW to generate and verify digital hardware cores. The examples cover the iob\_aoi, iob\_pulse\_gen, and iob\_soc cores, showcasing the automation of Verilog module generation from attributes.

\subsubsection{iob\_aoi Example}

The iob\_aoi core is a simple example that combines an AND, OR, and invert logic gates. The core's attributes are defined in the iob\_aoi.py file, available at \url{https://github.com/IObundle/py2hwsw/tree/main/py2hwsw/lib/hardware/basic_tests/iob_aoi}. To generate the iob\_aoi core, follow these steps:

\begin{enumerate}
\item Create or modify the iob\_aoi.py file to set the attributes of the core, describing how it should be generated using the Py2HWSW standard core dictionary interface.
\item Optionally, add more files to the setup directory as needed, such as manual Verilog sources or templates, scripts, or software.
\item Call the Py2HWSW setup process using the command:
\begin{lstlisting}[language=bash]
nix-shell --run "py2hwsw iob_aoi setup --build_dir '$(BUILD_DIR)'"
\end{lstlisting}
\item The generated build directory contains all the Verilog sources, Makefile, and configurations to run the core in various flows (simulation, FPGA).
\item To run the core in simulation, call the command: \texttt{make sim-run} from the build directory.
\end{enumerate}

\subsubsection{iob\_pulse\_gen Example}

The iob\_pulse\_gen core is used to generate wire pulses with configurable start and duration. The core's attributes are defined in the iob\_pulse\_gen.py file, available at \url{https://github.com/IObundle/py2hwsw/blob/main/py2hwsw/lib/hardware/clocks_resets/iob_pulse_gen/iob_pulse_gen.py}. To generate the iob\_pulse\_gen core, follow these steps:

\begin{enumerate}
\item Create or modify the iob\_pulse\_gen.py file to set the attributes of the core, describing how it should be generated using the Py2HWSW standard core dictionary interface.
\item Optionally, add more files to the setup directory as needed, such as manual Verilog sources or templates, scripts, or software.
\item Call the Py2HWSW setup process using the command:
\begin{lstlisting}[language=bash]
nix-shell --run "py2hwsw iob_pulse_gen setup --build_dir '$(BUILD_DIR)'"
\end{lstlisting}
\item The generated build directory contains all the Verilog sources, Makefile, and configurations to run the core in various flows (simulation, FPGA).
\item To run the core in simulation, call the command: \texttt{make sim-run} from the build directory.
\end{enumerate}

\subsubsection{iob\_soc Example}

The iob\_soc core is a more complex example used to create a system on chip. The core's attributes are defined in the iob\_soc.py file, available at \url{https://github.com/IObundle/iob-soc}. The iob\_soc.py file supports high-level Python parameters that allow configuring main SoC components like the CPU, memories, and peripherals. To generate the iob\_soc core, follow these steps:

\begin{enumerate}
\item Create or modify the iob\_soc.py file to set the attributes of the core, describing how it should be generated using the Py2HWSW standard core dictionary interface.
\item Optionally, add more files to the setup directory as needed, such as manual Verilog sources or templates, scripts, or software.
\item Call the Py2HWSW setup process using the command:
\begin{lstlisting}[language=bash]
nix-shell --run "py2hwsw iob_soc setup --build_dir '$(BUILD_DIR)'"
\end{lstlisting}
\item The generated build directory contains all the Verilog sources, Makefile, and configurations to run the core in various flows (simulation, FPGA).
\item To run the core in simulation, call the command: \texttt{make sim-run} from the build directory.
\end{enumerate}

These examples demonstrate the automation of Verilog module generation from attributes using Py2HWSW, showcasing the flexibility and ease of use of the framework.


\subsection{Customizing Py2HWSW}
\label{sec:customizing_py2}
% SPDX-FileCopyrightText: 2025 IObundle
%
% SPDX-License-Identifier: MIT

Py2HWSW allows users to customize its behavior and core generation process. When running cores directly from the cloned Py2HWSW repository, users can modify the cores or Py2HWSW scripts for debugging purposes.

The Py2HWSW repository contains several main folders, including lib, scripts, and generic folders.
\begin{verbatim}
.
└── py2hwsw
     ├── <py2 generic folders>
     ├── lib
     └── scripts
\end{verbatim}

The py2hwsw folder contains generic folders that are copied to every core build directory set up via Py2HWSW. These folders include standard Makefiles, FPGA board constraints, simulator dependencies, and other essential files.

To override the default files, users can create a file with the same name in their core's setup directory. For example, to override the default py2hwsw/document/tsrc/sim\_desc.tex, create a new document/tsrc/sim\_desc.tex in the core's setup directory. Py2HWSW will first copy the generic default file to the build directory and then copy the core files from the setup directory, overriding the default file.

The lib directory contains a library of cores provided by Py2HWSW. These cores are intended to be bug-free and do not typically require modifications. However, if users need to modify a core for their project, they can copy the corresponding core's setup directory to a subfolder in their project directory. Py2HWSW will use the first core it finds with the required name, so users can create custom modifications to the core specific to their project.

For example, to modify the iob\_and core, copy its folder from the Py2HWSW repository and place it in the user's project directory. When calling Py2HWSW from the project directory, it will find the copied iob\_and folder first and use it to generate the iob\_and core.

The scripts folder contains the Python scripts that make up the Py2HWSW tool. These scripts are typically only modified by developers, as they directly change the Py2HWSW program's behavior. However, users can modify these scripts for quick bug fixes or to add custom functionality.

By customizing Py2HWSW, users can tailor the tool to their specific needs and create custom cores and workflows. This flexibility allows users to adapt Py2HWSW to their project's requirements and create complex digital designs with ease.


\subsection{Troubleshooting}
\label{sec:troubleshooting}
% SPDX-FileCopyrightText: 2025 IObundle
%
% SPDX-License-Identifier: MIT

When encountering errors during the setup process with Py2HWSW, there are several steps you can take to diagnose and resolve the issue.

\subsubsection{Error Messages}

The main error message is usually printed in red color and provides information on where the issue originates, often due to a misconfiguration in the provided core dictionary. The traceback that follows is more useful for Py2HWSW developers, as it contains information on which Py2HWSW function has thrown the error.

\subsubsection{Debugging Options}

If more information is required to troubleshoot the issue, you can use the following options:

\begin{itemize}
\item The --debug\_level flag: When calling py2hwsw with the --debug\_level flag, you can print debug messages during the setup process. The higher the debug level, the more messages are printed.
\item Adding print statements: You can add print statements in your own core's .py file to understand when the script is being called and what contents it contains.
\item Modifying Py2HWSW scripts: Adding print statements to the Py2HWSW main scripts can also be useful, but this requires understanding the inner workings of Py2HWSW and is usually reserved for developers.
\end{itemize}

\subsubsection{Build Process Errors}

If the Py2HWSW setup process completes successfully, but the build process for a flow from the build directory gives errors (e.g., calling the Makefile from the build directory for simulation), follow these steps:

\begin{enumerate}
\item Check the generated Verilog sources: Verify that the contents of the generated Verilog sources are as intended.
\item Check tool-specific files: If the error message is simulator/tool-specific and does not seem related to the Verilog sources, check the constraints files, Makefiles, and other tool-specific files to determine where the issue originates.
\end{enumerate}

\subsubsection{Overriding Py2HWSW Generated Files}

If you need to modify a Py2HWSW generated file, you can override it by creating a new file with the same name in your core's setup directory. This allows you to customize the generated files to suit your specific needs.

By following these troubleshooting steps, you should be able to identify and resolve issues that arise during the setup and build process with Py2HWSW.


%
% Configuration
%
% \ifdefined\SECTIONCLEARPAGE
% \clearpage
% \fi
% \section{Configuration}
% \label{sec:config}
% \input{config}

% \subsection{Configuration Files}
% \label{sec:config_files}
% \input{config_files}

% \subsection{Command Line Options}
% \label{sec:command_line_options}
% \input{command_line_options}

% \subsection{Environment Variables}
% \label{sec:environment_vars}
% \input{environment_vars}

%
% Advanced Topics
%
% \ifdefined\SECTIONCLEARPAGE
% \clearpage
% \fi
% \section{Advanced Topics}
% \label{sec:advanced_topics}
% \input{advanced_topics}

% \subsection{Customizing the Framework}
% \label{sec:customizing_framework}
% \input{customizing_framework}

% \subsection{Creating Custom Classes}
% \label{sec:creating_custom_classes}
% \input{creating_custom_classes}


%
% Integrate Py2HWSW with external tools
%
% \section{Integrating with Other Tools}
% \label{sec:integrating_external_tools}
% \input{integrating_external_tools}

% \subsection{Integrate non-py2hwsw cores}
% \label{sec:non_py2_cores}
% \input{non_py2_cores}

% \subsection{Integrate py2hwsw cores in non-py2hwsw ecosystems}
% \label{sec:non_py2_ecosystems}
% \input{non_py2_ecosystems}

%
% Short Notation
%
% \section{Short Notation}
% \label{sec:short_notation}
% \input{short_notation}

%
% Appendices
%
% \ifdefined\SECTIONCLEARPAGE
% \clearpage
% \fi
% \section{Appendices}
% \label{sec:appendices}
% \input{appendices}

%\subsection{Glossary}
%\label{sec:glossary}
%\input{glossary}

%\subsection{Release Notes}
%\label{sec:release_notes}
%\input{release_notes}

%\subsection{Contributing}
%\label{sec:contributing}
%\input{contributing}

%\subsection{FAQ}
%\label{sec:faq}
%\input{faq}

%\subsection{Troubleshooting Guide}
%\label{sec:troubleshooting_guide}
%\input{troubleshooting_guide}

\end{document}
